\chapter{Personal Network Structures}
  % Justify why we're looking at this at all
  \section{Background}
    \subsection{What is social network analysis}
    \subsection{Personal networks in the US}
    \subsection{Personal networks in South Louisiana}
    \subsection{Language choice constrained by personal networks}
    \subsection{Linguistic variable constrained by network characteristics}
    \subsection{Previous work on relationships between personal networks and language in Louisiana}
  \section{Methods}
    \subsection{Data that I used}
    \subsection{Analyses carried out}
  \section{Results}

    % According to the GGS, mean age and education homophily in personal networks has not changed between 1985 and 2004 (J. A. Smith et al. 2014:439)
    % Occupational gender segregation has decreased steadily in the US between 1980 and 2003 (Tomaskovic-Devey et al. 2006, as cited in J. A. Smith et al. 2014:435)
    % According to the GGS, mean education homophily in personal networks has not changed between 1985 and 2004 (J. A. Smith et al. 2014:439)


