\chapter{Personal Network Structures}
  % Justify why we're looking at this at all
  \section{Background}
    \subsection{What is social network analysis}
      % Factor types that may impact the potential for out-group ties: demographic factors (e.g., immigration), institutional factors (e.g., work arrangements), economic factors (e.g., inequality), cultural factors (e.g., perceptions) J. A. Smith et al. (2014:437).
    \subsection{Social networks in the US}
      % It has been argued that network homophily can be an indirect measure for "the salience of sociodemographics features" in the sense that a feature that is not salient for a group of people is not likely to be something that motivates them to associate with each other (Blau & Schwartz 1984; Laumann 1966, both as cited in J. A. Smith et al. 2014:433)
      % Homophilous can in fact be beneficial in some ways, but one detriment to them is difficulty understanding others' world views (McPherson 2004 as cited in J. A. Smith et al. 2014:450)
      % According to the GGS, mean age and education homophily in personal networks has not changed between 1985 and 2004 (J. A. Smith et al. 2014:439)
        % Marital homogamy according to age and education has increased (Schwartz 2010, as cited in J. A. Smith et al. 2014:434)
      % Occupational gender segregation has decreased steadily in the US between 1980 and 2003 (Tomaskovic-Devey et al. 2006, as cited in J. A. Smith et al. 2014:435)
      % According to the GGS, mean education homophily in personal networks has not changed between 1985 and 2004 (J. A. Smith et al. 2014:439)
      % According to the GGS, mean religious homophily in personal networks has decreased (0.76 > 0.71) between 1985 and 2004 (J. A. Smith et al. 2014:439)
        % Marital homogamy according to religion has decreased (Fischer & Hout 2006, as cited in J. A. Smith et al. 2014:434)
      \subsubsection{Race and ethnicity}
        % Wright et al. (2014) argue that racial segregation is "a barometer of race relations" (174)
        % In 1985, racial and ethnic division was "the most salient social distinction structuring U.S. confidant relations" (Marsden 1988, as cited in J. A. Smith et al. 2014:446)
          % White-Asian ties have increased between 1985 and 2004 (J. A. Smith et al. 2014:446-447)
        % Racial homophily in the US has been very high and extremely stable for over 20 years as of 2014 (J. A. Smith et al. 2014:446)
          % According to the GSS, it decreased from 0.95 to 0.90 despite the country becoming more racially diverse (J. A. Smith et al. 2014:439)
          % Racial endogamy has declined over the 20th century but is still the most prevalent endogamous pattern in the US (Rosenfeld 2008)
        % Racial (and ethnic?) occupational segregation had "flattened out and changed slowly" between the 1980s and 2000s (Tomaskovic-Devey et al. 2006, as cited in J. A. Smith et al. 2014:436)
    \subsection{Social networks in South Louisiana}
      \subsubsection{Race and ethnicity}
        % Between the 1990 and 2010 censuses, there were fewer low-diversity White neighborhoods nationwide (66% > 42.5%), and Black dominant neighborhoods nationwide have become more racially diverse, though in Louisiana, low-diversity Black neighborhoods have actually increased (Wright et al. 2014:176-177)
        % Beggs et al. (1996) predicted rural communities would be more racially and ethnically homogenous than urban communities and found racial and ethnic homogeneity to be even more prevalent in Louisiana than the country in general (though not as a statistically significant finding)
    \subsection{Language choice constrained by personal networks}
    \subsection{Linguistic variable constrained by network characteristics}
    \subsection{Previous work on relationships between personal networks and language in Louisiana}
  \section{Methods}
    \subsection{Data that I used}
    \subsection{Analyses carried out}
  \section{Results}
  \section{Conclusion}
    % Greater contact between groups can decrease homophily and the salience of those previously homophilous characteristics (J. A. Smith et al. 2014:434)


