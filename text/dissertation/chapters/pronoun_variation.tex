\chapter{Subject Pronoun Variation in Louisiana French}
  \section{Subject pronoun system}
    % Historical pronoun systems
      % Creole in 19th century New Orleans
        % mo, to (2sg), li, yé (imp), nou, vou (2pl), yé (3pl) (Fortier 1884:107)
      % Origin of non-hexagonal pronoun forms?
        % An argument for the origin of yo in Haitian Creole is the disjunctive pronoun eux (Lefebvre 2001:383-384)
          % One could imagine the same arguments for ye
    % Modern day pronouns
      % Table of groups by pronoun types from studies that look at whole pronoun systems
        % Groups divided by geographical location > social factors
        % For each pronoun type, just give the most common variant, and provide attested alternatives in the text below
        % The text below is for relevant details that aren't apparent in the table
        % Sources
          % Rottet (1995:178/193/197/207)
          % Klinlger (2003 "Turn":206/213)
      % 1sg
        % Possible variants
          % Phonetically, the consonant in je has been documented as /ʒ/, /ʃ/, /z/, /s/, and /h/ (Carmichael & Gudmestad 2019:72)
            % Semi-speakers in Terrebonne-Lafourche used a much wider variety of phonetic variants of je than young and old fluent speakers (Carmichael & Gudmestad 2019:79)
          % Mo and mwen (both mostly) and also mon in Pointe Coupee (Klingler 2003 "Turn":206/213)
          % In Terrebonne-Lafourche, je, moi je, moi, and NULL are attested (Rottet 1995:178)
        % Structural variation
          % Phonological
            % Among Indians in Terrebonne-Lafourche, moi for 1sg was more likely before a vowel (Gudmestad & Carmichael 2022:18)
            % 1sg as [h] among Indians in Terrebonne-Lafourche more frequent before vowels (Rottet 1996, as cited in Gudmestad & Carmichael 2022:5)
            % Among Indians in Terrbonne-Lafourche, the clitic for 1sg is less likely to be produced if moi is also produced (by phonetic variants of the clitic, this is significant for [h] and [ʃ] but not [ʒ]) (Gudmestad & Carmichael 2022:14-15)
          % Among Indians in Terrbonne-Lafourche, 1sg with -er verbs as [ʒ] is disfavored, as null or [h] is the baseline, and as [ʃ] is favored (Gudmestad & Carmichael 2022:14-15)
          % Among Indians in Terrebonne-Lafourche, moi for 1sg was more likely if the referent had changed (Gudmestad & Carmichael 2022:18)
          % Among White Cajuns in St Landry, je or moi je were associated with avoir whereas moi was associated with être and other verbs (Dubois 2001, as cited in Gudmestad & Carmichael 2022:6)
            % The result is confusing as only those with weak exposure to French used moi by itself
        % Social variation
          % Ethnicity
            % For 1sg, null was the most common for Indian speakers of all levels in Terrebonne-Lafourche, sometimes with moi (i.e., moi by itself was possible) (Gudmestad & Carmichael 2022:11)
            % 1sg as [z] in Terrebonne-Lafourche may be indicative of Indian identity (Dajko 2009, as cited in Gudmestad & Carmichael 2022:5)
          % Gender
            % Among Indians in Terrbonne-Lafourche, for 1sg, males disfavored [h] but were neutral between null, [ʃ], and [ʒ] (Gudmestad & Carmichael 2022:14-15)
          % Proficiency
            % Among Indian in Terrebonne-Lafourche, moi with 1sg was more likely to be produced by semispeakers (Gudmestad & Carmichael 2022:18)
            % Among Indians in Terrbonne-Lafourche, for 1sg, semispeakers disfavored [h] but were neutral between null, [ʃ], and [ʒ] (Gudmestad & Carmichael 2022:14-15)
          % Social context
            % 1sg as [h] among Indians in Terrebonne-Lafourche more frequent and [z] less frequent in casual speech (Dajko 2009, as cited in Gudmestad & Carmichael 2022:5)
      % 2sg
        % to in Pointe Coupee (Klingler 2003 "Turn":206)
      % 2sg polite
        % T-V distinctions have been studied at least since R. Brown & Gilman (1960) who suggested that the was based on the power hierarchy between the two interlocutors (the nonreciprocal power semantic) and the intimacy vs formality of a relationship (the solidarity semantic)
        % Possible variants
          % vou (mostly), vo, and ou in Pointe Coupee (Klingler 2003 "Turn":206/213), but I suspect ou is simply lenition of [v] in vou
            % In Pointe Coupee, Klingler (2003 "Turn") suggests that this form is used with older people but also possibly women and Blacks addressing Whites (208)
          % In Terrebonne-Lafourche, tu is used in addition to vous (Rottet 1995:197)
        % Constraints
          % In Terrebonne-Lafourche, elderly participants variably claimed to have tutoi'd there parents or vouvoi'd them (Rottet 1995:186)
      % 3sg
        % Null subjects are attested Îles-de-la-Madeleine, Quebec for 3sg (LeBlanc 1996, as cited in Carmichael & Gudmestad 2019:83)
        % Possible variants
          % li in Pointe Coupee (Klingler 2003 "Turn":206)
          % Gender
            % li in Pointe Coupee is genderless (Klingler 2003 "Turn":170)
            % In Houma, Indians often used il without concern for gender whereas Cajuns also had elle (Dajko 2009, as cited in Dajko 2012:292)
        % Structural variation
          % Phonological
            % li in Pointe Coupee sometimes becomes simply [i] before a consonant (Klingler 2003 "Turn":210-211)
      % 1pl
        % Possible variants
          % Not a single token of nous was attested in Terrebonne-Lafourche (Rottet 1995:173)
          % nou(zòt) (mostly) but also rarely no in Ponte Coupee (Klingler 2003 "Turn":206/213)
      % 2pl
        % Possible variants
          % vou(zòt) (mostly) as well as zòt and zo in Pointe coupee (Klingler 2003 "Turn":206/213)
            % ouzò(t) is also attested, but I suspect this is just lenition of /v/
          % vous-autres, vous, and tu in Terrebonne-Lafourche (Rottet 1995:193)
      % 3pl
        % Null subjects are attested Îles-de-la-Madeleine, Quebec for 3pl (LeBlanc 1996, as cited in Carmichael & Gudmestad 2019:83)
        % Possible variants
          % Cajuns in the 1970s were already using ça for 3pl (Johnson 1976:31)
          % ye in Pointe Coupee and rarely ça (Klingler 2003 "Turn":206)
          % In Terrebonne-Lafourche, ils, ça, eux, and eux-autres are attested (Rottet 1995:207)
        % Structural variation
          % Semantics
            % Younger speakers who used only ça and eux in Terrebonne-Lafourche used the former for inanaminate referents and the latter for animate (Rottet 1995:219)
      % Variation in pronoun systems along racial or ethnic lines considered holistically
        % In Pointe Coupee Creole, Whites use pronouns that are closer to French and that agree in gender and/or number (e.g. avèk èl vs avèk li (f.)) (Klingler 2003 "Turn":117)
  \section{Methods}
    \subsection{Pronoun coding}
      % Verb type
        % This essentially is about defining the envelope of variation (Meyerhoff 2011, as cited in Kerswill & Watson 2014:139)
      % Location
      % Age
      % French background
      % Gender
      % Education
      % Occupational class
    \subsection{Analyses carried out}
  \section{Results}
  \section{Discussion}
    % The principle of multiple causes is the idea that more than one factor influences language variation patterns (Bayley 2002, as cited in Hazen 2014:12)
      % This does not only mean that multiple identified factors are at work but that unindentified factors may also be at work
    % Dying languages are expected to show 1) reduction/simplification, 2) stylistic shrinkage, 3) increased variability, and 4) rapid change (Dressler 1972; Dorian 1989; O'Shannessy 2011; Palosaari & Campbell 2011; Schmidt 1985; Wolfram 2004, all as cited in Carmichael & Gudmestad 2019:72)
    % Dorian (1994 "Varieties") has argued that variation is socially neutral in the sense that it exists but doesn't mean anything to speakers (as cited in Mayeux 2024:28)