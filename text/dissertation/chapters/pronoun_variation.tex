\chapter{Subject Pronoun Variation in Louisiana French}
  \section{Subject pronoun system}
    % Historical pronoun systems
      % Creole in 19th century New Orleans
        % mo, to (2sg), li, yé (imp), nou, vou (2pl), yé (3pl) (Fortier 1884:107)
      % Origin of non-hexagonal pronoun forms?
        % An argument for the origin of yo in Haitian Creole is the disjunctive pronoun eux (Lefebvre 2001:383-384)
          % One could imagine the same arguments for ye
    % Modern day attested forms
      % 1sg
        % Phonetically, the consonant in je has been documented as /ʒ/, /ʃ/, /z/, /s/, and /h/ (Carmichael & Gudmestad 2019:72)
          % Semi-speakers in Terrebonne-Lafourche used a much wider variety of phonetic variants of je than young and old fluent speakers (Carmichael & Gudmestad 2019:79)
        % Among White Cajuns in St Landry, je or moi je were associated with avoir whereas moi was associated with être and other verbs (Dubois 2001, as cited in Gudmestad & Carmichael 2022:6)
          % The result is confusing as only those with weak exposure to French used moi by itself
        % For 1sg, null was the most common for Indian speakers of all levels in Terrebonne-Lafourche, sometimes with moi (i.e., moi by itself was possible) (Gudmestad & Carmichael 2022:11)
        % 1sg as [z] in Terrebonne-Lafourche may be indicative of Indian identity (Dajko 2009, as cited in Gudmestad & Carmichael 2022:5)
        % Among Indians in Terrbonne-Lafourche, for 1sg, males disfavored [h] but were neutral between null, [ʃ], and [ʒ] (Gudmestad & Carmichael 2022:14-15)
        % Among Indians in Terrbonne-Lafourche, for 1sg, semispeakers disfavored [h] but were neutral between null, [ʃ], and [ʒ] (Gudmestad & Carmichael 2022:14-15)
        % Among Indian in Terrebonne-Lafourche, moi with 1sg was more likely to be produced by semispeakers (Gudmestad & Carmichael 2022:18)
        % 1sg as [h] among Indians in Terrebonne-Lafourche more frequent before vowels (Rottet 1996, as cited in Gudmestad & Carmichael 2022:5)
        % 1sg as [h] among Indians in Terrebonne-Lafourche more frequent in casual speech (Dajko 2009, as cited in Gudmestad & Carmichael 2022:5)
          % 1sg as [z] among Indians in Terrebonne-Lafourche less frequent in casual speech (Dajko 2009, as cited in Gudmestad & Carmichael 2022:5)
        % Among Indians in Terrbonne-Lafourche, 1sg with -er verbs as [ʒ] is disfavored, as null or [h] is the baseline, and as [ʃ] is favored (Gudmestad & Carmichael 2022:14-15)
        % Among Indians in Terrbonne-Lafourche, the clitic for 1sg is less likely to be produced if moi is also produced (by phonetic variants of the clitic, this is significant for [h] and [ʃ] but not [ʒ]) (Gudmestad & Carmichael 2022:14-15)
        % Among Indians in Terrebonne-Lafourche, moi for 1sg was more likely before a vowel or if the referent had changed (Gudmestad & Carmichael 2022:18)
      % 3sg
        % Null subjects are attested Îles-de-la-Madeleine, Quebec for 3sg and 3pl (LeBlanc 1996, as cited in Carmichael & Gudmestad 2019:83)
      % 3pl
        % Null subjects are attested Îles-de-la-Madeleine, Quebec for 3sg and 3pl (LeBlanc 1996, as cited in Carmichael & Gudmestad 2019:83)
  \section{Methods}
    \subsection{Pronoun coding}
      % Verb type
      % Location
      % Age
      % French background
      % Gender
      % Education
      % Occupational class
    \subsection{Analyses carried out}
  \section{Results}
  \section{Discussion}
    % Dying languages are expected to show 1) reduction/simplification, 2) stylistic shrinkage, 3) increased variability, and 4) rapid change (Dressler 1972; Dorian 1989; O'Shannessy 2011; Palosaari & Campbell 2011; Schmidt 1985; Wolfram 2004, all as cited in Carmichael & Gudmestad 2019:72)
    % Dorian (1994 "Varieties") has argued that variation is socially neutral in the sense that it exists but doesn't mean anything to speakers (as cited in Mayeux 2024:28)