\chapter{Subject Pronoun Variation in Louisiana French}
  \section{Background}
    \subsection{Subject pronouns in general}
      \subsubsection{Status of clitics}
    \subsection{Social and structural constraints on French and Creole} % Each section begins with a brief acknowledgement that these factors CAN be significant in language variation in general
      % General constraints that have been found that aren't necessarily analyzed here
        % 1sg as [h] among Indians in Terrebonne-Lafourche more frequent before vowels (Rottet 1996, as cited in Gudmestad & Carmichael 2022:5)
        % 1sg as [h] among Indians in Terrebonne-Lafourche more frequent in casual speech (Dajko 2009, as cited in Gudmestad & Carmichael 2022:5)
          % 1sg as [z] among Indians in Terrebonne-Lafourche less frequent in casual speech (Dajko 2009, as cited in Gudmestad & Carmichael 2022:5)
        % Among Indians in Terrbonne-Lafourche, 1sg with -er verbs as [ʒ] is disfavored, as null or [h] is the baseline, and as [ʃ] is favored (Gudmestad & Carmichael 2022:14-15)
        % Among Indians in Terrbonne-Lafourche, the clitic for 1sg is less likely to be produced if moi is also produced (by phonetic variants of the clitic, this is significant for [h] and [ʃ] but not [ʒ]) (Gudmestad & Carmichael 2022:14-15)
        % Among Indians in Terrebonne-Lafourche, moi for 1sg was more likely before a vowel or if the referent had changed (Gudmestad & Carmichael 2022:18)
      \subsubsection{Verb type}
        % Among White Cajuns in St Landry, je or moi je were associated with avoir whereas moi was associated with être and other verbs (Dubois 2001, as cited in Gudmestad & Carmichael 2022:6)
          % The result is confusing as only those with weak exposure to French used moi by itself
      \subsubsection{Location}
      \subsubsection{Age}
      \subsubsection{French background}
        % Among Indians in Terrbonne-Lafourche, for 1sg, semispeakers disfavored [h] but were neutral between null, [ʃ], and [ʒ] (Gudmestad & Carmichael 2022:14-15)
         % Among Indian in Terrebonne-Lafourche, moi with 1sg was more likely to be produced by semispeakers (Gudmestad & Carmichael 2022:18)
      \subsubsection{Gender}
        % Among Indians in Terrbonne-Lafourche, for 1sg, males disfavored [h] but were neutral between null, [ʃ], and [ʒ] (Gudmestad & Carmichael 2022:14-15)
      \subsubsection{Education}
      \subsubsection{Socioeconomic class}
  \section{Methods}
    \subsection{Data that I used}
    \subsection{Analyses carried out}
  \section{Results}