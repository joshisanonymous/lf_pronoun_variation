\chapter{Introduction}
  % What is Louisiana, as a whole, in the public imagination
  % Why do we care about Louisiana
  % Objective: Examine how ethnicity and race are or are not expressed through subject pronouns in French and Creole as spoken in Louisiana
  \section{Fractal recursivity}
    % Definition
      % One of three semiotic practices used to "locate, interpret, and rationalize sociolinguistic complexity, identifying linguistic varieties with 'typical' persons and activities" (Irvine & Gal:36-37)
      % Specifically: a semiotic practice of distinction in which people project a salient opposition from one level to another, such as urban vs surburban into a smaller local context such as a school (Irvine & Gal 2000:38)
      % As a case of fractal recursivity, Irvine & Gal (2000) describe colonial Europeans imagining the sociolinguistic situation in Africa as being analogous to Europe's when mapping out the languages of the continent (49-55).
    % Previous work
  \section{Race and ethnicity}
    % Are these different concepts to begin with, and are they different from each other?
      % Ethnicity has been defined as "an aspect of a person's social identity that also forms part of an individual's self concept" that is based on heritage and subsumes race (i.e., race is one form that this aspect takes) (Phinney 1992, as cited in Susberry 2004:21)
    % Bi-/multi-racial identities
      % The number of bi- and multi-racial people may be underestimated when tabulated as 1) people may be unaware of their own ancestry, 2) may identify with only one race, or 3) 
      % A biracial person has been defined as "someone whose parents are of two socially and phenotypically distinct racial backgrounds" (Root 1992, as cited in Susberry 2004:16)
    \subsection{Race and ethnicity in the United States}
      % The basis of racial labeling in the US
        % It has been argued that race in the US isn't entirely about phenotype as, for example, there are lighter skinned Black people who are still considered Black because of things like ancestry and culture (Susberry 2004:1)
      % Racial and ethnic diversity situation
        % According to census records, the US has gone from 1 in 8 people being non-White to 1 in 4 (or 1 in 3 if Latinxs are treated as non-White) (Fischer & Hout 2006:25-26)
          % Ethnic and racial diversity, measured with Shannon entropy using census records, has increased between 1990 and 2010 (0.4576 > 0.6015) (Wright et al. 2014:175)
          % The pattern holds through the 2020 census except for non-Whites increasing to 42% of the population when treating Latinxs as non-White (Census 2020)
          % "[R]oughly half" of White Americans thought they were a minority already by 2000 (Alba et al. 2005, as cited in J. A. Smith et al. 2014:437)
        % Latinx is currently the largest non-White group (19.5%) followed by Blacks (13.7%) (Census 2020)
          % Growth in the Latinx population in 2011 was driven more by births than by immigration (Pew Hispanic Center 2011, as cited in Wright et al. 2014:181)
        % According to censuses, between 1990 and 2010, White-dominant neighborhoods decreased from 66.0% to 42.5% (i.e., diversity increased) (Wright et al. 2014:176)
        % Most immigrants settled in highly populous metropolitan areas according to the 2000 and 2010 censuses (75%) (Wright et al. 2014:175)
      % Racial and ethnic tolerance and inequality
        % Attitudes have moved towards greater tolerance of racial and ethnic minorities from the 1970s to the 2010s (Bobo et al. 2012; Firebaugh & Davis 1988, both as cited in J. A. Smith et al. 2014:437)
        % Racial income inequality decreased earlier in the 20th century but stalled by 2000 (Leicht 2008, as cited in J. A. Smith et al. 2014:436)
    \subsection{Race in Louisiana}
      % Historically
        % Quadroon was a label used for people shortly after the Louisiana Purchase who were of one fourth African descent (J. Martin 2000:57; Susberry 2004:9)
        % Octoroon was a label used for people shortly after the Louisiana Purchase who were of one eighth African descent (Susberry 2004:9)
        % Gens de couleur had more social status than Blacks (enslaved people) but were still not afforded the same status as Whites (Barthelemy 2000; Dunbar-Nelson 2000, both as cited in Susberry 2004:10-11; J. Martin 2000:60)
        % In the 20th century, Creoles of Color who resented being classified as Black would either try to pass as White if possible or simply assert that they were racially Creole rather than Black, though many Creoles of Color simply aligned themselves with Blacks (Susberry 2004:12-13)
      % General make-up today
        % The increase in the Latinx-dominant neighborhoods in the US has mostly occurred in the western states (i.e., not Louisiana), according to census records between 1990 and 2010 (Wright et al. 2014:179)
    \subsection{Cajun identity}
      % Study participants in St Landry identifying as Cajun (N = 9) all also identified as White (Klingler 2003 "Language":82)
    \subsection{Creole identity}
      % It has been suggested that 4 factors have helped maintain an interest in Creole identity up to the present: 1) A concern that Louisiana exports being marketed as "Cajun" will erase Creole contributions, 2) the lack of laws or even census options that would pigeonhole Creoles into identifying as Black instead, 3) the focus on French language education, and 4) the popularity of geneological research (Susberry 2004:14)
      % Study participants in St Landry identifying as Creole (N = 19) all also identified as Black (Klingler 2003 "Language":82)
    \subsection{Language ideologies accounting for race or ethnicity}
      % Study participants in St Landry claimed to speak French if they identified as Cajun and Creole if they identified as Creole, regardless of whether their speech appeared structurally more like French or more like Creole (Klingler 2003 "Language"), but Louisiana isn't the only place where race and ethnicity influence language ideologies.
      % During the Quiet Revolution in Quebec, one might hurl an insult at people speaking French in public by demanding that they "speak White" (i.e., English) instead (Lamarre 2014:149)
    \subsection{Racially conditioned language variation}
    \subsection{Ethnically conditioned language variation}
  \section{French and Creole}
    % Glossonym
      % Creole speakers in Pointe Coupee (determined by the language structure) mostly call their language "créole" but sometimes also "français" or "cadien", though even when using the latter two glossonyms, speakers note that their language isn't like French from France or Cajun from Lafayette (Klingler 2003 "Turn":128)
    % Where is situated socially and geographically?
    \subsection{Status in Louisiana}
      % Where did French come from in the first place
        % Although Acadian French is at least implicitly considered the main origin of French in rural Louisiana, there are many sources for French (Klingler 2009)
        % Gudmestad & Carmichael (2022) had a number of study participants (Indians) who were native speakers in their 30s (in 2007-2008) in Terrebonne-Lafourche (6-7)
      % Where did Creole come from in the first place
        % People who were enslaved and transported from Senagambia mainly spoke Malinke/Maninka, Serer, Wolof, or Pulaar (Klingler 2003 "Turn":57)
        % African languages may be responsible features such as (Klingler 2003 "Turn":62-66):
          % Predicate clefting in which an adjective is repeated in its usual location (e.g., Se malad li malad)
          % Preverbal markers
          % Frequent copula absence
          % Postnominal definite and demonstrative determiners
          % The plural postnominal definite article being identical to the 3pl subject pronoun
    \subsection{French vs Creole}
      % French and French-related varieties that exist
        % "Cajun French" (though he prefers "Louisiana Regional French" as it is not spoken purely by Cajuns), "Louisiana Creole", and "Colonial French" (which he and Picone (1998) prefer to call Plantation Society French since it developed in the 19th century) (Klingler 2003 "Language":77)
      % Speakers' approaches to naming their languages as one or the other
        % It's not unusual to place languages under a label based on cultural traits of the speakers rather than the structure of their ways of speaking
          % Linguists once described Cangin as a variety of Serer because Wolofs in the area saw the cultural practices of both Cangin and Sereer speakers as being the same (Irvine & Gal 2000:57)
        % When asked plainly (without pushing for greater specificity), speakers of both varieties in St Landry will call their language "French" (Klingler 2003 "Language":78-79)
          % Apparently Spitzer (1977) and Le Menestrel (1999) have also noted this labeling approach by speakers.
      % Structural blurring of boundaries between French and Creole
        % This is particularly prevalent in Lafayette, Breaux Bridge, and Vacherie (Klingler 2003 "Language":78), the former two being where much of my data comes from.
        % Klingler (2003 "Language") uses 1sg subject pronouns, past perfective constructions, and the verb 'to have' as diagnostics for structurally disentangling French and Creole (80)
          % 1sg was a better option than other pronouns, such as 3sg, because in the latter case, the pronouns il and li can be reduced in both French and Creole, respectively, to [i] (Klingler 2003 "Language":80)
            % This is notable because it inadvertantly gives us another example of the difficulty in drawing a structural boundary between French and Creole
    \subsection{Distinguishing features}
      % Geographic variation
        % Almost all of the variants marked as Creole (by Klingler, 1sg subject pronouns, perfective aspect representation, and the verb 'to have') used by speakers in St Landry Parish were produced by 4 speakers from Leonville, Prairie Ville, and Arnaudville (Klingler 2003 "Language":83).
          % These towns are all along Bayou Teche which historically was lined with plantations (Klingler 2003 "Language":83)
          % Most variants were marked as French in general
          % Most Creole variants were also used by Blacks, though not exclusively (some White Cajuns used Creole-marked perfective aspect)
        % Almost all variants of 1sg pronouns, perfective aspect, and the verb 'to have' used by speakers in Pointe Coupee (3 Black, 2 White) were variants marked as Creole save for some tokens (~7%) of 'avoir' used only by Whites (Klingler 2003 "Language":81)
          % For 1sg specifically, only 2 tokens were not mo, and those 2 tokens were null rather than je (Klingler 2003 "Language":81)
    \subsection{Subject pronoun system}
      % Terrebonne-Lafourche
        % For 1sg, null was the most common for Indian speakers of all levels, sometimes with moi (i.e., moi by itself was possible) (Gudmestad & Carmichael 2022:11)
      % Not Terrebonne-Lafourche
    \subsection{Racially and ethnically conditioned pronouns} % Add general social and structural constraints to pronoun predictors chapter
      % 1sg as [z] in Terrebonne-Lafourche may be indicative of Indian identity (Dajko 2009, as cited in Gudmestad & Carmichael 2022:5)