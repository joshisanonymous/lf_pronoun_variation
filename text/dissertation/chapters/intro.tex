\chapter{Introduction}
  % What is Louisiana, as a whole, in the public imagination
  % Why do we care about Louisiana
  % Objective: Examine how ethnicity and race are or are not expressed through subject pronouns in French and Creole as spoken in Louisiana
  \section{Race and ethnicity}
    % Are these different concepts to begin with, and are they different from each other?
      % Definitions of ethnicity
        % "[A]n aspect of a person's social identity that also forms part of an individual's self concept" that is based on heritage and subsumes race (i.e., race is one form that this aspect takes) (Phinney 1992, as cited in Susberry 2004:21)
      % Definitions of race
        % "'[A] concept which signifies and symbolizes social conflicts and interests by referring to different types of human bodies'" (Omi & Winant 1994, as cited in Fought 2013:388)
        % A social construct with biological, underpinnings that is hierarchical and ethnicity as a social construct underpinned by "sociocultural factors that are neglected by the biological definitions of race" (Susberry 2004:6-7)
    % Bi-/multi-racial identities
      % A biracial person has been defined as "someone whose parents are of two socially and phenotypically distinct racial backgrounds" (Root 1992, as cited in Susberry 2004:16)
        % This definition uses "parents" as a buffer between the person in question and their ancestors, as there is likely a socially defined limit to what people will consider in terms of what counts as someone's "racial background"
          % There is 24% European ancestry in African-American genomes (Bryc et al. 2015:42)
            % But that doesn't mean that African-Americans by and large identify as biracial
          % My own participants [especially or mostly Creoles?] discussed how common mixed racial ancestries are among both Creoles and Cajuns, usually in terms of introducing Indians into the picture, but how this is also not also something people want to acknowledge
            % Indians themselves are a group that has been described as being conceptualized by US society as a single race despite having a wide range of phenotypical features (Bonilla-Silva 1997:472)
      % Biracial identity may follow its own development trajectory, and several models have been developed for this:
        % Kerwin & Ponterotto's (1995) model: 1) Growing recognition of distinctions in appearance (up to 5 yo), 2) use of descriptive language to define themselves and their families (early school), 3) appearance, culture, and language recognized as group membership criteria (pre-adolescence), and then 4) express knowledge of groups based on race, ethnicity and/or relgion (not just appearance)
        % Poston's (1990) model: 1) Personal identity (self-esteem is based on achievements), 2) choice of group categorization (into one race), 3) enmeshment/denial (related to having only chosen one race), 4) appreciation (of their multiple races), and then 5) integration (of all their racial identities) (as cited in Susberry 2004:19)
        % Kich's (1992) model: 1) dissonance/awareness of differentness (3-10 yo), 2) struggle for acceptance (8-adolescence), and finally 3) self-acceptance and assertion as biracial (late adolescence-adulthood) (as cited in Susberry 2004:18)
      % Factors influencing how bi-/multi-racial people self-identify
        % It has been found that exposure to a "context in which being Biracial is a meaningful concept" leads to border identities (i.e., single label identities that acknowledge biraciality), that a White appearance leads to transendent identities (i.e., rejection of racial labels), and that having dark skin and growing up in a Black environment leads to Black identities Brunsma & Rockquemore (2001, as cited in Susberry 2004:28)
      % The number of bi- and multi-racial people may be underestimated when tabulated as 1) people may be unaware of their own ancestry, 2) may identify with only one race, or 3) may adhere to the concept of situational ethnicity (Susberry 2004:3)
        % On the 2000 US census, 5.5% of people chose Other for race and 2.4% Multiracial (Singer 2002, as cited in Susberry 2004:2)
          % This was also the first year the census allowed people to check off more than one race (US Census Bureau 2000, as cited in Susberry 2004:2)
        % As Holliday (2019) pointed out, the ease of access sociolinguists have to racial census data combined with the inflexibility of race options on the census has been a source of erasure of racial identities that do not neatly fit into one box or another (2)
      % It's important to not simplify and erase bi-/multi-racial identities, but additionally, as pointed out by Holliday (2016), one must be careful not to essentialize them as being somehow special or not normal in comparison to monoracial individuals (24-27)
    \subsection{In the United States}
      % The basis of racial labeling in the US
        % It has been argued that race in the US isn't entirely about phenotype as, for example, there are lighter skinned Black people who are still considered Black because of things like ancestry and culture (Susberry 2004:1)
      % Racial and ethnic diversity situation
        % According to census records, the US has gone from 1 in 8 people being non-White to 1 in 4 (or 1 in 3 if Latinxs are treated as non-White) (Fischer & Hout 2006:25-26)
          % Ethnic and racial diversity, measured with Shannon entropy using census records, has increased between 1990 and 2010 (0.4576 > 0.6015) (Wright et al. 2014:175)
          % The pattern holds through the 2020 census except for non-Whites increasing to 42% of the population when treating Latinxs as non-White (Census 2020)
          % "[R]oughly half" of White Americans thought they were a minority already by 2000 (Alba et al. 2005, as cited in J. A. Smith et al. 2014:437)
        % Latinx is currently the largest non-White group (19.5%) followed by Blacks (13.7%) (Census 2020)
          % Growth in the Latinx population in 2011 was driven more by births than by immigration (Pew Hispanic Center 2011, as cited in Wright et al. 2014:181)
        % According to censuses, between 1990 and 2010, White-dominant neighborhoods decreased from 66.0% to 42.5% (i.e., diversity increased) (Wright et al. 2014:176)
        % Most immigrants settled in highly populous metropolitan areas according to the 2000 and 2010 censuses (75%) (Wright et al. 2014:175)
        % In 1990, only 6% of Blacks and Whites married outside of their race, but 30% of Asians and Hispanics and 67% of American Indians did (Water 2000, as cited in Susberry 2004:3)
      % Racial and ethnic tolerance and inequality
        % Attitudes have moved towards greater tolerance of racial and ethnic minorities from the 1970s to the 2010s (Bobo et al. 2012; Firebaugh & Davis 1988, both as cited in J. A. Smith et al. 2014:437)
          % Despite this, there is still a long way to go
            % Life outcomes
              % Racial income inequality decreased earlier in the 20th century but stalled by 2000 (Leicht 2008, as cited in J. A. Smith et al. 2014:436)
              % 71% of the African-American youths in Van Hofwegen & Wolfram's (2010) longitudinal study (1990-2006) were below the poverty line (431)
            % Historically, Black-White biracial Americans were viewed by Whites simply as Black, leading to enslavement and later continued discrimination (Khanna 2011, as cited in Holliday 2016:17-18)
              % There is evidence of this still being true today, 
          % Linguistic minstrelsy involves imitating minorities' speech for one's own profit (Bucholtz & Lopez 2011, as cited in Eberhardt & Freeman 2015:304)
            % Iggy Azalea (Eberhardt & Freeman 2015) and White rappers in general do this (C. Cutler 2002, as cited in C. Cutler 2007:520)
            % This suggests that tolerance has perhaps dovetailed into exploitation
    \subsection{Racially conditioned language variation}
      % Center this discussion on interpretations of the significance of such variation
      % It has been argued that in studies of African-Americans using linguistic features indexing White identities, researchers conceptualize this as assimilation, but in studies of White people using African-American features (e.g., Bucholtz 1999), researchers conceptualize this as indexing African-American qualities that these speakers admire (Eckert 2008 "Where":27)
        % White people "can and do pass for native speakers of AAE" (C. Cutler 2002, as cited in C. Cutler 2007:529)
      % Becker (2014) looked at several linguistic variables in the Lower East Side, comparing between how they've changed for African-Americans vs non-African-Americans:
        % Rhoticity indexes regional identity: AAs have remained non-rhotic, and non-AAs have become rhotic (52)
        % The BOUGHT vowel indexes localness: AAs have remained raised where others have lowered (49-50)
        % Lisa has deleted copulas when the topic is New York rather than African-Americanness (48)
      % Even when race is given a more fine-grained conceptualization, it has proven to be meaningful for language variation
        % Black-White biracial men in DC were found to have different intonational patterns depending on their self-identifications singular Black, using the pattern traditionally found to be associated with Black identities in general, versus self-identifications as border, using a pattern that has not traditionally been found to be indexical of Black identities (Holliday 2019:8)
      % Implicating creole languages
        % In the South Carolina Sea Islands, a pair of speakers, one White and one Black, both were documented using (English-based) creole phonological features but only the Black speaker used creole morphosyntactic features (Rickford 1985, as cited in Fought 2013:390-391)
      % There is indeed work on non-Blacks/African-Americans, as well, but probably not as much, and it more often is characterized as work on ethnicity rather than race in these cases
        % In New York, Asian and non-Asian speakers differed in their realizations of /ɛ/ and breathiness with Asian men producing more fronted variants and breathier speech (Newman & Wu 2011:166/168)
          % Newman & Wu (2011) suggest plainly that these features index an Asian racial identity (and importantly not a national heritage identity as the Asian participants Chinese and Korean) (171)
    \subsection{Ethnically conditioned language variation}
      % A Lumbee Indian being interviewed by a African-American Cherokee showed how ethnic identity can be linguistically negotiated in real time as the speakers' speech was most different when discussing race and most similar while discussing friends and family (Schilling-Estes 2004)
      % White students and Latinx students in California were seen using different variants of /æ/, but variants used within the Latinx group also index their place in the social order of the school, making them available for non-Latinxs to use, as well (Eckert 2008 "Where")
      % Something that's apparent in many of these studies on variation between ethnic groups is that race is often conflated with ethnicity, suggesting that the two are difficult to disentangle in the US
        % Indeed, the in some of these cases, these are only being discussed under the current heading because the researchers involved framed these as studies of ethnicity
    \subsection{Race in Louisiana}
      % Historically
        % The Louisiana Purchase was in 1803 (Klingler 2003 "Turn":107)
        % Quadroon was a label used for people shortly after the Louisiana Purchase who were of one fourth African descent (J. Martin 2000:57; Susberry 2004:9)
        % Octoroon was a label used for people shortly after the Louisiana Purchase who were of one eighth African descent (Susberry 2004:9)
        % Plaçage was a system used in the New World in which White Europeans would establish liaisons with quadroon and octoroon women -- women who weren't allowed to legally marry -- that led to children, in which the women participated in order to gradually obtain lighter skin and higher social status (Bryan 2000; Dunbar-Nelson 2000, T. McNeill 1994, all as cited in Susberry 2004:9; J. Martin 2000:57-58; Klingler 2003 "Turn":15-16)
          % The Black Code gave these children the status of their mothers, thus they were initially enslaved upon birth (Bryan 2000, as cited in Susberry 2004:9)
        % Gens de couleur had more social status than Blacks (enslaved people) but were still not afforded the same status as Whites (Barthelemy 2000; Dunbar-Nelson 2000, both as cited in Susberry 2004:10-11; J. Martin 2000:60)
        % For the time between the Louisiana Purchase and the Civil War, francophone Louisianians also attempted to unify around culture and language to enact laws that would prevent incoming "Americans" from participating in local politics (Johnson 1976:20)
          % This perhaps suggest a downplaying of race, though it's unclear what sources Johnson uses for this or how extensive it really was, and as I note elsewhere, there is evidence that race was still an important social organizing factor among Louisianians
          % Post-Civil War, it wasn't until 1971 that a French-speaking governor was elected: Edwin Edwards (Johnson 1976:29)
      % General make-up today
        % The increase in the Latinx-dominant neighborhoods in the US has mostly occurred in the western states (i.e., not Louisiana), according to census records between 1990 and 2010 (Wright et al. 2014:179)
        % It has been argued that ethnicity has become "polarized around race" in Louisiana (Dajko 2012:290)
          % Not only have Cajuns and Creoles been defined by residents according to physical traits but also Indians (Dajko 2007, as cited in Dajko 2012:279-280)
    \subsection{Creole identity}
      % It has been suggested that 4 factors have helped maintain an interest in Creole identity up to the present: 1) A concern that Louisiana exports being marketed as "Cajun" will erase Creole contributions, 2) the lack of laws or even census options that would pigeonhole Creoles into identifying as Black instead, 3) the focus on French language education, and 4) the popularity of geneological research (Susberry 2004:14)
      % For Creoles, Susberry (2004) divides their historical development as a group into 4 stages: 1) dissonance, 2) struggle for validation, 3) negotiation of group identity, and 4) activism (8)
        % At each stage, factors influencing Creoles' self-conceptualizations included 1) physical appearance, 2) low social status of Blacks, 3) negative treatment by Whites and Blacks, and 4) members' consciousness of the group's sociopolitical evolution (Susberry 2004:8)
      % In mid- to late-19th century New Orleans, Creole was still such a general term that it was applied to food as a synonym for local (Johnson 1976:21)
        % But for people, the term has been applied to varies groups, such as 1) White Louisianians of French and/or Spanish descent, 2) people of African descent, or 3) people of mixed African and European descent (Dajko 2012:289)
      % Up to the Louisiana Purchase
        % Around the time of the Louisiana Purchase, Creoles were White descendants of Europeans born in the Americas (Kein 2000, as cited in Susberry 2004:7-8)
          % Fortier (1884) says French and Spanish specifically rather than Europeans (98)
      % After the Louisiana Purchase
        % After the Louisiana Purchase, Creole eventually came to be the name for "mixed race people of color" (Martin 2000, as cited in Susberry 2004:7-8)
          % Although White Creoles, at least in New Orleans, "occupied a high standing" (Fortier 1884:98)
        % Johnson (1976) suggests that, in the 19th century, a "creole Negro" was an enslaved person who spoke French and sometimes "the free black French-speaking population" (25)
          % He doesn't give a source for this, and as the work of others has suggested, this is probably an oversimplification
        % White Americans and White Creoles were already attempting to socially collapse Free People of Color in with enslaved Blacks by 1830, though these efforts weren't very successful (Daniel 1992; Gehman 1994; Bryan 2000, all as cited in Susberry 2004:10-11)
      % After the Civil War
        % After the Civil War, poverty (Gehman 1994) and the abolishment of slavery became factors that made it easier for Whites to collapse Creoles of Color into the same social category as Blacks (previously enslaved people) (Susberry 2004:11)
          % In the 20th century, Creoles of Color themselves tended to align themselves with Blacks between the Civil War and the Civil Rights Movement despite seeing themselves as culturally distinct (Dubois & Melançon 2000; Gehman 1994, both as cited in Susberry 2004:12)
          % For Creoles of Color who resented being classified as Black would either try to pass as White if possible or simply assert that they were racially Creole rather than Black (Susberry 2004:12-13)
        % As for White Creoles
          % They began calling themselves Cajun in 1977 (Trépanier 1988, 1991, as cited in Dajko 2012:289)
      % Some definitions used by researchers today
        % "Multiracial people whose ancestral history is connected to Louisiana" (Susberry 2004:8)
        % Mixed race people with French cultural origins (Giancarlo 2019:34), though she argues that a common thread in thread in researchers' definitions is that Creoles are necessarily not Cajuns
    \subsection{Cajun identity}
      % Cajuns were historically seen as racially "ambiguous", "not black, but seen as less white than the Anglo-Saxons" (Tentchoff 1980; Walton 2003, both as cited in Giancarlo 2019:32)
        % Indeed, Cajuns began owning enslaved people in the 1770s and most did by 1810 (Brasseaux 1985), most being as much as 68% (Baker 1974, both as cited in Giancarlo 2019:31)
        % When "Americans" arrived, they described Cajuns as poor, rural, White francophones (Brasseaux, as cited in Dajko 2012:289)
      % Their origin myth is heavily focused on the Grand dérangement
        % The Acadians were expelled from Nova Scotia in 1755 (B. Brown 1986:399; Klingler 2003 "Turn":18-19; Neumann 1985:13)
          % Acadians themselves have been said to arrive in Canada from Poitou, Brittany, Normandy, Aunis, Saintonge, and Angoumois (Brasseaux 2005, as cited in Dajko 2012:281-282;  Massignon 1962, as cited in Rottet 1995:93; R. A. Brown 1988:16; Klingler 2003 "Turn":19)
          % They initially settled in The Acadian/German Coast (along the Mississippi in St James Parish), Opelousas, and the Attakapas (Klingler 2003 "Turn":100; Neumann 1985:13; Fortier 1884:99; Brasseaux 1987, as cited in Dajko 2012:284)
            % There numbers have been estimated to be between 1,500 and 10,000 (Ficatier 1957, as cited in Neumann 1985:9; Brasseaux 1992; Leblanc 1979; Taylor 1984; the latter three as cited in Rottet 1995:96; Klingler 2009, as cited in Dajko 2012:283)
        % There are reasons to believe that the Acadian origin myth is not the full story, such as a lack of numbers 
          % During the first half of the 19th century along Bayou Lafourche, French-speaking immigrants outnumbered Acadians by 3 to 2 (Brasseaux 1992, as cited in Dajko 2012:285)
      % Cajun membership
        % Non-racial criteria
          % Dajko (2012) suggests one can establish membership through 1) foodways, 2) music, 3) language (French), 4) "and until recently, the Catholic faith", which she also applies as valid criteria for Creole membership (280)
      % Giancarlo (2019) argues that characterizing all White South Louisianians as Cajun "minimizes a diversity of white ethnicities" (33)
        % She seems to mean that White South Louisianians might otherwise maintain different ethnic identities rather than that the racial identities aligned with Cajun identity
        % Cajuns weren't necessarily good at forming sociopolitical alliances with other groups
          % In the 19th century, Cajuns were Catholic like many New Orleanians but didn't form sociopolitical alliances with them due to a "urban-rural antipathy," and they were rural like northern Louisianians but didn't form alliances because Northerners were "rabid anti-Catholic Protestants" (Johnson 1976:27)
        % Other ethnic groups (Scots-Irish, Italians, Germans, some Native Americans) have been incorporated in the Cajun ethnicity, so ancestry and last names have been argued to not be necessary conditions for Cajun identity (Brasseaux 1992; Stanford 2016, both as cited in Giancarlo 2019:33)
    \subsection{Language ideologies accounting for race or ethnicity}
      % Examples outside of Louisiana
        % There is a long history of racializing languages
          % In the colonial French Caribbean, scholars such as Saint-Méry (1797) linked creole languages to Creole people and thus valorized creole languages as somewhere between the substrates of enslaved people and the lexifier of the slavers (Aboh & deGraff 2017:4-5)
        % During the Quiet Revolution in Quebec, one might hurl an insult at people speaking French in public by demanding that they "speak White" (i.e., English) instead (Lamarre 2014:149)
        % In San Antonio, Ecuador, one informant associated Quichua with Indianness, rural poverty, and femininity but Spanish with urban progress and masculinity (Rindstedt & Aronsson 2002:737)
      % The situation in Louisiana
        % Race and ethnicity have been intertwined with language ideologies in Louisiana for a long time
          % Fortier (1884) described White Creoles in New Orleans before the Civil War as speaking "very good French" (98)
            % Creoles in the 19th century and earlier (i.e., Creoles of Color and New World-born Europeans) aren't really associated with Louisiana Creole (Dajko 2012:290)
          % Johnson (1976) was still describing an estimated 5,000 out of all the "New Orleans creoles" as speaking "French", and indeed he gives examples that are marked for French in terms of pronouns and determiners (i.e., il me tarde and le steering wheel), but also notes the existence of "many English words" in their speech (25), suggesting perhaps that it would no longer be considered "very good French"
            % However, Johnson likely doesn't mean White Creoles as Fortier meant when using the term "New Orleans creoles", as he explains that, post-Civil War in New Orleans, the elite Blacks spoke a "New Orleans form of" French, but the rest of the Black population spoke "a dialect" called "creole patois", "creole dialect", "Negro French", or "gumbo French" (Johnson 1976:26)
        % Study participants in St Landry claimed to speak French if they identified as Cajun and Creole if they identified as Creole, regardless of whether their speech appeared structurally more like French or more like Creole (Klingler 2003 "Language")
  \section{Status of French and Creole in Louisiana today}
    % The importation of French into Louisiana is mostly simple as Louisiana was a French colony, but what happened after isn't as simple
      % Although Acadian French is at least implicitly considered the main origin of French in rural Louisiana, there are many sources for French (Klingler 2009)
        % Indeed, there was sustained contact with France and immigration from France all the way up to the Civil War (Brasseaux 2005; Picone 1997, both as cited in Dajko 2012:280)
        % Johnson (1976) describes all francophones in Louisiana as descending from four groups: 1) L'ancienne population (those from France who settled in New Orleans and along the Mississippi starting in 1718), 2) les français étrangers (those from France that came later), 3) Cajuns (those from Nova Scotia starting in 1755), and 4) those from Santo Domingo starting in 1791 (19)
          % He doesn't actually give sources for these
          % The original French settlers came from 6 provinces: 1) Île-de-France, 2) Brittany (Bretagne), 3) Champagne, 4) Poitou, 5) Aunis, and 6) Burgundy (Bourgogne) (Brasseaux 1987, as cited in Dajko 2012:281)
          % During Spanish control from 1762 to 1800, most French immigrants came from Guyenne and Provence (Brasseaux 2005, as cited in Dajko 2012:282)
          % The bulk of Saint-Domingue migrants arrived in 1809 and 1810 (Klingler 2003 "Turn":22/79) and settled in New Orleans (Debien & Le Gardeur 1981, as cited in Klingler 2003 "Turn":22)
        % Fortier (1884) described the French of Louisiana as having a "purity" due to men of rich families almost always being educated in France "during the old regime" (99)
          % But despite this being a general description of French in the state, it's not a description of Acadian French, suggesting the presence of other French-speaking groups
            % Fortier (1884) complained that the Acadians who arrived "did not contribute toward keeping the French language in a state of purity" (99)
              % Some examples Fortier (1884) gives of the "not very elegant" French of the Acadians: j'avions et j'étions, paré for prêt, il mouille for il pleut (99)
      % In the 19th century, French, Creole, and English were all spoken widely
        % Enough so that the Louisiana legislature purportedly had appointed interpreters (Gayarré, as cited in Fortier 1884:97-98)
          % In 1812, English was in fact the official language of the legislature, but work continued in both English and French up until it was abolished in 1916 (Johnson 1976:29/37-38)
          % French language newspapers existed
            % Le Moniteur de la Louisiane (1794 by Louis Duclot) was the first, L'Abeille (until 1925, "most famous"), L'Ami des Lois, Le Courrier de la Louisiane, and Le Propagateur Catholique were important, and Le Courrier de la Nouvelle-Orléans (1902-1955, though it struggled) was the last exclusively in French (Johnson 1976:22)
            % These stopped being published in Pointe Coupee in 1872 (Klingler 2003 "Turn":108)
            % Even as late as the 1970s, the bilingual Acadiana Profile was launched (Johnson 1976:29), but this was clearly a niche item at that point
        % As I mentioned elsewhere, there were efforts to unify around language between the Louisiana Purchase and Civil War to resist the influence of "American" immigrants (Johnson 1976:20)
      % Early 20th century
        % Numbers
          % Johnson (1976) describes Cajuns in particular as being rather isolated in the late 19th and early 20th centuries in comparison with other francophone populations, resulting in their greater continued use of French (28)
            % Even by Johnson's time, it was estimated that 60% of Cajuns lived in rural areas (Bobo & Charlton 1974, as cited in Johnson 1976:27)
            % 30% lived in poverty still compared to 10.7% nationally, and the median number of years of schooling completed by those over 25 years old was 8.6 compared to 9.2 for Louisiana in general and 12.2 nationally (either the 1970 census or Bobo & Charlton 1974, as cited in Johnson 1976:27-28)
            % Likewise, Pointe Coupee Creole was in part maintained due to 19th century residents being insular to the point of marrying cousins as a norm (Klingler 2003 "Turn":109)
          % The decline became rapid after 1900 in New Orleans and especially after WWI (Johnson 1976:22), though he gives no sources for this
            % The last holdouts were unemployed wives up until WWI (Johnson 1976:22)
            % Elsewhere the decline has not been nearly as rapid
          % On the 1970 census, 572,262 Louisianians reported French as their mothertongue (as cited in Johnson 1976:19/36 and in Neumann 1985:16)
            % CODOFIL itself estimated much higher around this time: 1.5 million (Johnson 1976:36)
          % Among all children in Evangeline Parish and Pointe Coupee Parish in 1960, it was estimated that, as children, 38% spoke French, 53% spoke English, and 9% spoke both (Bertrand & Beale 1965, as cited in Johnson 1976:32)
            % Among Black children in Evangeline Parish and Pointe Coupee Parish in 1960, it was estimated that 31% had a French background and 69% an English (Bertrand & Beale 1965, as cited in Johnson 1976:33)
        % Cultural indicators
          % It has been reported generally that Cajuns had taken "great pride" in their language by the 1970s, hence the establishment of CODOFIL in 1968 (Johnson 1976:28)
          % CODOFIL was bringing in an estimated 200 foreign francophone teachers each year to teach roughly 40,000 students (Johnson 1976:29)
            % Importantly, these teachers were using textbooks from France (Johnson 1976:28)
            % Much of CODOFIL's efforts have been focused on school immersion programs
              % It has been argued that, for language revitalization, focusing on schools is helpful but has a limited impact (Hornberger & King 1996, as cited in Rindstedt & Aronsson 2002:723)
          % Shortly after the establishment of CODOFIL in 1968, the Louisiana legislature launched Télévision-Louisiane as a non-profit French-language television broadcasting corporation (Johnson 1976:29)
          % As I mentioned elsewhere, the bilingual newspaper the Acadiana Profile began being published in the 1970s (Johnson 1976:29)
          % As I mentioned elsewhere, Edwin Edwards, in 1971, became the first French-speaking governor elected since the Civil War (Johnson 1976:29)
      % More recently
        % Projections
          % Johnson (1976) predicted, using crude methods, that native French-speaking families in Louisiana would cease to exist by 2010 (34-35)
        % Current numerical indications
          % Gudmestad & Carmichael (2022) had a number of study participants (Indians) who were native speakers in their 30s (in 2007-2008) in Terrebonne-Lafourche (6-7)
    \subsection{Defining a language as a Creole}
      % There have been attempts to define creoles as distinctive due to shared structural features
        % McWhorter (2011) offered 4 linguistic features that would also indicate that creoles develop from pidgins, but at least Aboh & deGraff (2017) reject this (14)
          % Aboh & deGraff (2017), based on many examples, instead argue that the linguistic structure these languages take on is always based on the languages that are in contact (16)
        % This often means arguing that creoles are simpler than other languages
          % It has been argued as a counter-example that Haitian Creole NPs are more complex than both French and Gbe NPs (Aboh & deGraff 2017:12)
      % But what makes a creole is the sociohistorical circumstances of its development
        % Mintz (1971) suggested that creoles only come about if the right conditions are met for demographics, social relationships, and the community setting (481)
        % In the colonial period, slavery was much more prevalent in the south (39.8% of pop) than the the middle colonies (7.0%) and New England (3.1%) (Franklin & Moss 1988, as cited in Rickford 1997:320)
          % Bickert (1981) has argued that as much as 80% of the population needs to be substrate speakers for a creole to emerge (as cited in Rickford 1997:317)
    \subsection{Historical development of Creole in Louisiana}
      % Where did Creole come from in the first place
        % Slavery
          % There were initially 5,310 enslaved people brought to Louisiana in the early 1700s from The Senegal Concession (Sierra Leone in the south to Mauritania in the north), the Gulf of Benin, and Angola (Cabinda specifically) (Klingler 2003 "Turn":6)
          % When the Spanish took over Louisiana in 1762 up until 1800, the number of enslaved people in Louisiana increased from ~4,600 to ~24,000 (Hall 1992, as cited in Klingler 2003 "Turn":20)
            % Of the roughly 55% of those whose origins are know during this time, they were 55% African, 37% Creole (born in the Louisiana), and 4% from English-speaking regions (Klingler 2003 "Turn":20-21)
          % In 1788 in Pointe Coupee, enslaved people outnumbered free people by about 3 to 1 (Klingler 2003 "Turn":103)
          % The foreign importation of enslaved people was banned in 1804 (Klingler 2003 "Turn":67)
            % Enslaved people could still be moved from elsewhere in the US to Louisiana
          % By 1860, 76% of the large plantations (more than 50 enslaved people) in Pointe Coupee were owned by "Americans" (Costello 1999, as cited in Klingler 2003 "Turn":107)
        % People who were enslaved and transported from Senagambia mainly spoke Malinke/Maninka, Serer, Wolof, or Pulaar (Klingler 2003 "Turn":57)
        % African languages may be responsible features such as (Klingler 2003 "Turn":62-66):
          % Predicate clefting in which an adjective is repeated in its usual location (e.g., Se malad li malad)
          % Preverbal markers
          % Frequent copula absence
          % Postnominal definite and demonstrative determiners
          % The plural postnominal definite article being identical to the 3pl subject pronoun
        % Who speaks it today generally
          % In Pointe Coupee, more Whites, because the influx of English speaking slaves brought to the area in the 19th century meant that black slaves often learned English before their white owners (Klingler 2003 "Turn":108)
    \subsection{French vs Creole}
      % French and Creole varieties that exist
        % "Cajun French" (though he prefers "Louisiana Regional French" as it is not spoken purely by Cajuns), "Louisiana Creole", and "Colonial French" (which he and Picone (1998) prefer to call Plantation Society French since it developed in the 19th century) (Klingler 2003 "Language":77)
        % Colonial French/Plantation Society French, Louisiana Creole, and Cajun French/Louisiana Regional French (Dajko 2012:280-281)
      % Speakers' approaches to naming their languages as one or the other
        % It's not unusual to place languages under a label based on cultural traits of the speakers rather than the structure of their ways of speaking
          % Linguists once described Cangin as a variety of Serer because Wolofs in the area saw the cultural practices of both Cangin and Sereer speakers as being the same (Irvine & Gal 2000:57)
        % When asked plainly (without pushing for greater specificity), speakers of both varieties in St Landry will call their language "French" (Klingler 2003 "Language":78-79)
          % Apparently Spitzer (1977) and Le Menestrel (1999) have also noted this labeling approach by speakers.
        % Creole speakers in Pointe Coupee had "a clear sense of belonging to a broader world of people who speak 'French'" (Klingler 2003 "Turn":128)
          % Creole speakers in Pointe Coupee (determined by the language structure) mostly call their language "créole" but sometimes also "français" or "cadien", though even when using the latter two glossonyms, speakers note that their language isn't like French from France or Cajun from Lafayette (Klingler 2003 "Turn":128)
            % Indeed, Bertrand & Beale (1965) characterized Black children from Pointe Coupee as being from English or "French" backgrounds (as cited in Johnson 1976:33) rather than Creole backgrounds, yet Klingler by the 1990s was describing speakers in this parish as using Creole features almost to the exclusion of French marked variants
      % Regardless of naming approaches, there is evidence of mutual intelligibility between all these ways of speaking
        % Creole speakers in Pointe Coupee spoke of being able to successfully communicate with French speakers from both inside of and outside of Louisiana, although some French speakers have described comprehension as difficult (Klingler 2003 "Turn":128)
      % The issue of maintaining two discrete social constructs
        % Even in accepted monolingual situations, it's important to distinguish between the languages of individuals and community languages
          % Pierrehumbert (2012) acknowledged that we can conceptualize lexicons at either the individual or the community level (173)
        % In some cases, creoles have been considered distinct from their lexifiers based on their abrupt formation creating a genetic break between the two (e.g., Thomason & Kaufman 1988, as cited in Aboh & deGraff 2017:8-9) and/or the proposal of the criteria that genetic relatedness requires an unbroken succession of native speakers (e.g., Labov 2007; Ringe et al. 2002, both as cited in Aboh & deGraf 2017:25)
          % I would argue that the idea of genetic relationships between languages is a social construct to begin with and so the existence of a break is a sociopolitical debate, but what is clear empirical fact is that linguistic features have diffused from speakers of the lexifier language to speakers of the substrates
        % For others, maintaining two discrete social constructs is an issue of how to maintain, valorize, and in some cases, revitalize creoles
          % This has been an issue for AAL
            % Baugh (2015) was emphatic that AAVE is English (766-767)
      % Structural blurring of boundaries between French and Creole
        % Even in studies of bilingualism, there is debate over the degree to which separate "languages" have shared versus distinct grammars
          % The BIA+ model of language processing argues for shared lexicons (Dijkstra & Van Heuven 2002)
          % The input switch theory of language processing argues that bilinguals process language by activating the required language and deactivating the other (Macnamara & Kushnir 1971, as cited in Spivey & Marian 1999:281)
            % This was not supported by Spivey & Marian's (1999) experiment with Russian-English bilinguals (283)
        % This is particularly prevalent in Lafayette, Breaux Bridge, and Vacherie (Klingler 2003 "Language":78), the former two being where much of my data comes from.
        % Klingler (2003 "Language") uses 1sg subject pronouns, past perfective constructions, and the verb 'to have' as diagnostics for structurally disentangling French and Creole (80)
          % 1sg was a better option than other pronouns, such as 3sg, because in the latter case, the pronouns il and li can be reduced in both French and Creole, respectively, to [i] (Klingler 2003 "Language":80)
            % This is notable because it inadvertantly gives us another example of the difficulty in drawing a structural boundary between French and Creole
        % Analogously, Aboh & deGraff (2017) argued for a genetic relationship between French and Haitian Creole as a number of preverbal markers "have straightforward etyma in French morphemes in periphrastic verbal constructions whose meanings often overlap with those of the corresponding TMA+V combinations in" (10-11)
          % (It may be worth introducing their examples here as a number of the markers are basically identical to those in Louisiana Creole
            % Aboh & deGraff (2017) discuss te, ap, the verbal suffix -e, future a(va), fini, sòt, and dwe (10)
              % Detges (2000) also cites uses of être après reported by Pomier (1835) (152)
              % Detges (2000) links the Mauritian Creole future marker pou to the use of être pour to express the near future in 1880 French (145)
          % Many of the markers of Louisiana Creole can likewise be interpreted as more or less still corresponding semantically and functionally with their French correlates, as I will discuss in the distinguishing features section
        % On a sociopolitical level, these similarities in structure can be obscured in particular via choices in how to represent creoles orthographically
      % Spelling convention used in this study
        % It has been argued for British Creole that non-standard spelling functions both as a way to represent a language that has no orthographic standard as well as a way to distance itself from its superstrate when words could be spelled the same but aren't, such as <Jameka> or <kool> (Sebba 1998, as cited in Androutsopoulos 2000:515)
        % Bengali and Hindi speakers who mix English into their online writing in the former languages sometimes write the former languages using the Roman script for convenience (Barman et al. 2014:13)
          % Likewise, writing here using French orthographic conventions is not a stance on the primacy of French but rather a matter of convenience
    \subsection{General distinguishing features of French and Creole}
      % "Markers"
        % Perhaps a New Orleans variety of Creole
          % Alé for future, sra for the future anterior, té for past imperfective, sré for conditional, apé for present [probably means progressive] (Fortier 1884:108)
        % Habitual past is expressed with se for some speakers in Pointe Coupee (Klingler 2003 "Turn":261)
        % Future
          % The simple future is expressed with "a" in Pointe Coupee Creole (Klingler 2003 "Turn":258-259)
          % Ale is common in St Tammany Creole (Klingler 2003 "Turn":260)
        % Progressive
          % e (mostly), ap, and ape in Pointe Coupee (Klingler 2003 "Turn":255)
        % Conditional
          % se in Pointe Coupee Creole (Klingler 2003 "Turn":261)
        % ka(pa(b)) is presented as a verb in Pointe Coupee Creole (Klingler 2003 "Turn":247) and as simultaneously an auxiliary verb and a marker in Breaux Bridge Creole (Neumann 1985:223)
          % Neumann (1985) also include pe 'peut' as equivalently an auxiliary and a marker, though for both kapab and pe, she draws the connection with être capable de and pouvoir plainly (223)
          % In Pointe Coupee, pas comes after pe but before kapab (Klingler 2003 "Turn":275)
            % Although this is also a difference between ve and oler, respectively (Klingler 2003 "Turn":276)
        % Perfect aspect expressed with bin [bɪn] in Pointe Coupee (Klingler 2003 "Turn":262-263)
        % Syntactic order
          % te + pa + e in Pointe Coupee (Klingler 2003 "Turn":26/258)
          % te + bin + pas in Pointe Coupee (Klingler 2003 "Turn":263)
          % Relative to verbs, pa comes before those expressing habitual present and before those expressing any other tense+aspect, regardless of short or long form (Klingler 2003 "Turn":320-321)
            % Some verbs have pa invariably preceding them in Pointe Coupee: bezòn, fini, gen, kapab, konne, kouri, ole (Klingler 2003 "Turn":322)
          % With other markers, pa comes after all markers except ape and ale in Pointe Coupee (Klingler 2003 "Turn":26)
      % Possessive forms such as noukenne (Fortier 1884:107)
        % Also attested in Pointe Coupee (Klingler 2003 "Turn":187-188)
        % These are only described in grammars of what is labeled Creole, but speakers who have otherwise almost none of the other features from those grammars certainly produce this form, as well
          % e.g., YouTube interview with Canray Fontenot, my interview with Patrick Patin (I think)
      % Noun agglutinations
        % l-, n-, and z- can vary even within the speech of a single speaker in Pointe Coupee Creole (Klingler 2003 "Turn":160)
        % Agglutinations are dropped when the noun is preceded by a preposition in fixed expressions (Klingler 2003 "Turn":162)
      % Post-posed articles
        % 19th century Creole in New Orleans
          % -la (sg), -layé (pl) (definite), -cila (sg), -cilayé (pl) (demonstrative) (Fortier 1884:105-106)
        % Pointe Coupee Creole speakers produced -la (sg) and -yé (pl) for definite determiners (Klingler 2003 "Turn":65/84/172)
        % St Tammany speakers produce -le (pl) definite determiners (Klingler 2003 "Turn":84)
        % Pointe Coupee Creole has -sa-la (sg) and -sa-ye (pl) for a demonstrative determiner and sometimes -sa (Klingler 2003 "Turn":65/181-182)
        % These aren't exclusive, as les is produced, as well
          % In Pointe Coupee, les generally comes before plural person nouns and time expressions (Klingler 2003 "Turn":174-175)
          % In Pointe Coupee, les is attested as being combined with -sa-ye (Klingler 2003 "Turn":182) and in Breaux Bridge with les + -sa-la(-la) is the most common form for the demonstrative (Klingler 2003 "Turn":183)
          % Similarly, in Pointe Coupee, des is sometimes produced as a plural indefinite article (Klingler 2003 "Turn":172)
          % In my own data, les appears to be the default, and instead -yé only occurs in possessive constructions, presumably because les is replaced by a possessive determiner, and possessive determiners for these speakers aren't marked for number
          % These cases are simply more examples of the blurring of the boundary between what one would call French vs Creole
      % Morphology
        % Number
          % Didn't exist in Pointe Coupee Creole (Klingler 2003 "Turn":170)
        % Gender
          % Didn't exist for some in Pointe Coupee Creole except for cases where there is a semantic contrast related to the referent in Pointe Coupee Creole (cousin vs cousine) (Klingler 2003 "Turn":170)
        % Verbs
          % Two stems exist in Pointe Coupee Creole, but the systematicity of these is not clear for all speakers (Klingler 2003 "Turn":236-237)
          % Copulas can be left out
            % In equative questions (what's your name) and equative affirmative declaratative sentences where the subject is a pronoun in Pointe Coupee Creole (Klingler 2003 "Turn":292/331)
          % The verb kouri can be used to mean to go in Pointe Coupee Creole (Klingler 2003 "Turn":246)
          % ina as an existential expression
            % Present tense variants in Pointe Coupee: ena, y ena, na (Klingler 2003 "Turn":307-308)
              % The y ena version helps support my approach of treating ina a y a with an /n/ liaison rather than il a with the same liaison
            % For past, enave/inave/nave in Pointe Coupee Creole (Klingler 2003 "Turn":308)
      % Prepositions
        % From (i.e., out of) was expressed in Pointe Coupee Creole with dans, en-dans, or NULL (Klingler 2003 "Turn":356/360)
      % English influence
        % Two Cajun women (73 and 88) from Gonzales produced code-switching in an interview (B. Brown 1986:401)
          % She characterized these women as speaking English and "Cajun"
          % She argues the English influence is on both a lexical and structural level (400)
          % Types: 54 emblematic vs 28 intimate, meaning most switches were single nouns, and 82 conversational vs 3 situational, meaning switches didn't involve changes in participants or subjects (403)
            % Thge emblematic switches often involve modern technologies (B. Brown 1986:400-401)
        % French also has influenced the English spoken in Louisiana, for example B. Brown's (1986) two participants produced phrases like "he went in France" and "I love gumbo, cher" (400)
          % Dubois & Horvath's (2003) study of "Cajun Vernacular English" had men producing phrases such as "Me I went to the store" and "I've been married with my wife during twenty years" (37)
      % Geographic variation
        % Almost all of the variants marked as Creole (by Klingler, 1sg subject pronouns, perfective aspect representation, and the verb 'to have') used by speakers in St Landry Parish were produced by 4 speakers from Leonville, Prairie Ville, and Arnaudville (Klingler 2003 "Language":83).
          % These towns are all along Bayou Teche which historically was lined with plantations (Klingler 2003 "Language":83)
          % Most variants were marked as French in general
          % Most Creole variants were also used by Blacks, though not exclusively (some White Cajuns used Creole-marked perfective aspect)
        % Almost all variants of 1sg pronouns, perfective aspect, and the verb 'to have' used by speakers in Pointe Coupee (3 Black, 2 White) were variants marked as Creole save for some tokens (~7%) of 'avoir' used only by Whites (Klingler 2003 "Language":81)
          % For 1sg specifically, only 2 tokens were not mo, and those 2 tokens were null rather than je (Klingler 2003 "Language":81)
      % Ethnic variation
        % Johnson (1976) claimed that Cajuns in the 1970s pronounced <oi> pronounced as [wɛ] and had distinct lexical features such as amarrer, haler, asteur, and English borrowings (31-32)
          % It's not clear why he mentions English borrowings for Cajuns as he does mention this for New Orleans Creoles, as well
      % Racial variation
        % Several features differ along racial lines in Pointe Coupee Creole
          % Whites tend to pronounce rounded front vowels more often, whereas the rounding is more regularly lost among Blacks (Klingler 2003 "Turn":116)
          % Whites tend to pronounce word final and preconsonantal rs more regularly. Whites also use the affricate /tʃ/ in place of /k/ more often (Klingler 2003 "Turn":116)
          % Whites use gender and number agreement for adjectives and determiners much more often (e.g., se zekla vs so zekla-ye and ma manman vs mo manman and mo (f.) movèz vs mo move) (Klingler 2003 "Turn":116)
          % Whites use determiners that are closer to French and that agree in gender and/or number (e.g. lœrtchèn vwazinaj vs yekèn vwazinaj and tou le paròl-sa-la vs tou paròl-sa-ye) (Klingler 2003 "Turn":117)
          % Whites tend to use d(e) to indicate a relation between nouns where blacks will just juxtapose the nouns (e.g. enn mezon-d-ekòl vs enn lamezon-ekòl for schoolhouse) (Klingler 2003 "Turn":117)
          % Whites sometimes pronounce the past tense marker te and the clause final copula ye as ete and e, respectively (Klingler 2003 "Turn":119)
          % Whites tend to use a stem without the final e (or equivalent) in the habitual present and the 2nd person singular imperative, whereas blacks tend to use one long stem invariably in all contexts (Klingler 2003 "Turn":117)
          % Irregular verbs like to have, to want, and to can take French-like conjugations in the speech of Whites but less in the speech of Blacks (e.g. nou u vs nou te gen and tou me katen ave vs tou mo katen-ye te gen and cochon-la voule vs cochon-la t'ole and ye peu pa parle vs ye pa kapab pale and mo poura chante vs mo sa kapab chante) (Klingler 2003 "Turn":118)