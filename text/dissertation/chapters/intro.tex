\chapter{Introduction}
  % What is Louisiana, as a whole, in the public imagination
  % Why do we care about Louisiana
  % Objective: Examine how ethnicity and race are or are not expressed through subject pronouns in French and Creole as spoken in Louisiana
  \section{Fractal recursivity}
    % Definition
      % One of three semiotic practices used to "locate, interpret, and rationalize sociolinguistic complexity, identifying linguistic varieties with 'typical' persons and activities" (Irvine & Gal:36-37)
      % Specifically: a semiotic practice of distinction in which people project a salient opposition from one level to another, such as urban vs surburban into a smaller local context such as a school (Irvine & Gal 2000:38)
      % As a case of fractal recursivity, Irvine & Gal (2000) describe colonial Europeans imagining the sociolinguistic situation in Africa as being analogous to Europe's when mapping out the languages of the continent (49-55).
    % Previous work
  \section{Race and ethnicity}
    % Are these different concepts to begin with, and are they different from each other?
      % Ethnicity has been defined as "an aspect of a person's social identity that also forms part of an individual's self concept" that is based on heritage and subsumes race (i.e., race is one form that this aspect takes) (Phinney 1992, as cited in Susberry 2004:21)
      % Race defined as a social construct with biological, underpinnings that is hierarchical and ethnicity as a social construct underpinned by "sociocultural factors that are neglected by the biological definitions of race" (Susberry 2004:6-7)
    % Bi-/multi-racial identities
      % A biracial person has been defined as "someone whose parents are of two socially and phenotypically distinct racial backgrounds" (Root 1992, as cited in Susberry 2004:16)
        % This definition uses "parents" as a buffer between the person in question and their ancestors, as there is likely a socially defined limit to what people will consider in terms of what counts as someone's "racial background"
          % There is 24% European ancestry in African-American genomes (Bryc et al. 2015:42)
            % But that doesn't mean that African-Americans by and large identify as biracial
          % My own participants [especially or mostly Creoles?] discussed how common mixed racial ancestries are among both Creoles and Cajuns, usually in terms of introducing Indians into the picture, but how this is also not also something people want to acknowledge
      % Biracial identity may follow its own development trajectory, and several models have been developed for this:
        % Kerwin & Ponterotto's (1995) model: 1) Growing recognition of distinctions in appearance (up to 5 yo), 2) use of descriptive language to define themselves and their families (early school), 3) appearance, culture, and language recognized as group membership criteria (pre-adolescence), and then 4) express knowledge of groups based on race, ethnicity and/or relgion (not just appearance)
        % Poston's (1990) model: 1) Personal identity (self-esteem is based on achievements), 2) choice of group categorization (into one race), 3) enmeshment/denial (related to having only chosen one race), 4) appreciation (of their multiple races), and then 5) integration (of all their racial identities) (as cited in Susberry 2004:19)
        % Kich's (1992) model: 1) dissonance/awareness of differentness (3-10 yo), 2) struggle for acceptance (8-adolescence), and finally 3) self-acceptance and assertion as biracial (late adolescence-adulthood) (as cited in Susberry 2004:18)
      % Factors influencing how bi-/multi-racial people self-identify
        % It has been found that exposure to a "context in which being Biracial is a meaningful concept" leads to border identities (i.e., single label identities that acknowledge biraciality), that a White appearance leads to transendent identities (i.e., rejection of racial labels), and that having dark skin and growing up in a Black environment leads to Black identities Brunsma & Rockquemore (2001, as cited in Susberry 2004:28)
      % The number of bi- and multi-racial people may be underestimated when tabulated as 1) people may be unaware of their own ancestry, 2) may identify with only one race, or 3) may adhere to the concept of situational ethnicity (Susberry 2004:3)
        % On the 2000 US census, 5.5% of people chose Other for race and 2.4% Multiracial (Singer 2002, as cited in Susberry 2004:2)
          % This was also the first year the census allowed people to check off more than one race (US Census Bureau 2000, as cited in Susberry 2004:2)
        % As Holliday (2019) pointed out, the ease of access sociolinguists have to racial census data combined with the inflexibility of race options on the census has been a source of erasure of racial identities that do not neatly fit into one box or another (2)
      % It's important to not simplify and erase bi-/multi-racial identities, but additionally, as pointed out by Holliday (2016), one must be careful not to essentialize them as being somehow special or not normal in comparison to monoracial individuals (24-27)
    \subsection{Race and ethnicity in the United States}
      % The basis of racial labeling in the US
        % It has been argued that race in the US isn't entirely about phenotype as, for example, there are lighter skinned Black people who are still considered Black because of things like ancestry and culture (Susberry 2004:1)
      % Racial and ethnic diversity situation
        % According to census records, the US has gone from 1 in 8 people being non-White to 1 in 4 (or 1 in 3 if Latinxs are treated as non-White) (Fischer & Hout 2006:25-26)
          % Ethnic and racial diversity, measured with Shannon entropy using census records, has increased between 1990 and 2010 (0.4576 > 0.6015) (Wright et al. 2014:175)
          % The pattern holds through the 2020 census except for non-Whites increasing to 42% of the population when treating Latinxs as non-White (Census 2020)
          % "[R]oughly half" of White Americans thought they were a minority already by 2000 (Alba et al. 2005, as cited in J. A. Smith et al. 2014:437)
        % Latinx is currently the largest non-White group (19.5%) followed by Blacks (13.7%) (Census 2020)
          % Growth in the Latinx population in 2011 was driven more by births than by immigration (Pew Hispanic Center 2011, as cited in Wright et al. 2014:181)
        % According to censuses, between 1990 and 2010, White-dominant neighborhoods decreased from 66.0% to 42.5% (i.e., diversity increased) (Wright et al. 2014:176)
        % Most immigrants settled in highly populous metropolitan areas according to the 2000 and 2010 censuses (75%) (Wright et al. 2014:175)
        % In 1990, only 6% of Blacks and Whites married outside of their race, but 30% of Asians and Hispanics and 67% of American Indians did (Water 2000, as cited in Susberry 2004:3)
      % Racial and ethnic tolerance and inequality
        % Attitudes have moved towards greater tolerance of racial and ethnic minorities from the 1970s to the 2010s (Bobo et al. 2012; Firebaugh & Davis 1988, both as cited in J. A. Smith et al. 2014:437)
          % Despite this, there is still a long way to go
            % Life outcomes
              % Racial income inequality decreased earlier in the 20th century but stalled by 2000 (Leicht 2008, as cited in J. A. Smith et al. 2014:436)
              % 71% of the African-American youths in Van Hofwegen & Wolfram's (2010) longitudinal study (1990-2006) were below the poverty line (431)
            % Historically, Black-White biracial Americans were viewed by Whites simply as Black, leading to enslavement and later continued discrimination (Khanna 2011, as cited in Holliday 2016:17-18)
              % There is evidence of this still being true today, 
          % Linguistic minstrelsy involves imitating minorities' speech for one's own profit (Bucholtz & Lopez 2011, as cited in Eberhardt & Freeman 2015:304)
            % Iggy Azalea (Eberhardt & Freeman 2015) and White rappers in general do this (C. Cutler 2002, as cited in C. Cutler 2007:520)
            % This suggests that tolerance has perhaps dovetailed into exploitation
    \subsection{Racially conditioned language variation}
      % Center this discussion on interpretations of the significance of such variation
      % It has been argued that in studies of African-Americans using linguistic features indexing White identities, researchers conceptualize this as assimilation, but in studies of White people using African-American features (e.g., Bucholtz 1999), researchers conceptualize this as indexing African-American qualities that these speakers admire (Eckert 2008 "Where":27)
        % White people "can and do pass for native speakers of AAE" (C. Cutler 2002, as cited in C. Cutler 2007:529)
      % Becker (2014) looked at several linguistic variables in the Lower East Side, comparing between how they've changed for African-Americans vs non-African-Americans:
        % Rhoticity indexes regional identity: AAs have remained non-rhotic, and non-AAs have become rhotic (52)
        % The BOUGHT vowel indexes localness: AAs have remained raised where others have lowered (49-50)
        % Lisa has deleted copulas when the topic is New York rather than African-Americanness (48)
      % Even when race is given a more fine-grained conceptualization, it has proven to be meaningful for language variation
        % Black-White biracial men in DC were found to have different intonational patterns depending on their self-identifications singular Black, using the pattern traditionally found to be associated with Black identities in general, versus self-identifications as border, using a pattern that has not traditionally been found to be indexical of Black identities (Holliday 2019:8)
      % There is indeed work on non-Blacks/African-Americans, as well, but probably not as much, and it more often is characterized as work on ethnicity rather than race in these cases
        % In New York, Asian and non-Asian speakers differed in their realizations of /ɛ/ and breathiness with Asian men producing more fronted variants and breathier speech (Newman & Wu 2011:166/168)
          % Newman & Wu (2011) suggest plainly that these features index an Asian racial identity (and importantly not a national heritage identity as the Asian participants Chinese and Korean) (171)
    \subsection{Ethnically conditioned language variation}
      % A Lumbee Indian being interviewed by a African-American Cherokee showed how ethnic identity can be linguistically negotiated in real time as the speakers' speech was most different when discussing race and most similar while discussing friends and family (Schilling-Estes 2004)
      % White students and Latinx students in California were seen using different variants of /æ/, but variants used within the Latinx group also index their place in the social order of the school, making them available for non-Latinxs to use, as well (Eckert 2008 "Where")
      % Something that's apparent in many of these studies on variation between ethnic groups is that race is often conflated with ethnicity, suggesting that the two are difficult to disentangle in the US
        % Indeed, the in some of these cases, these are only being discussed under the current heading because the researchers involved framed these as studies of ethnicity
    \subsection{Race in Louisiana}
      % Historically
        % Quadroon was a label used for people shortly after the Louisiana Purchase who were of one fourth African descent (J. Martin 2000:57; Susberry 2004:9)
        % Octoroon was a label used for people shortly after the Louisiana Purchase who were of one eighth African descent (Susberry 2004:9)
        % Plaçage was a system used in the New World in which White Europeans would establish liaisons with quadroon and octoroon women -- women who weren't allowed to legally marry -- that led to children, in which the women participated in order to gradually obtain lighter skin and higher social status (Bryan 2000; Dunbar-Nelson 2000, T. McNeill 1994, all as cited in Susberry 2004:9; J. Martin 2000:57-58)
          % The Black Code gave these children the status of their mothers, thus they were initially enslaved upon birth (Bryan 2000, as cited in Susberry 2004:9)
        % Gens de couleur had more social status than Blacks (enslaved people) but were still not afforded the same status as Whites (Barthelemy 2000; Dunbar-Nelson 2000, both as cited in Susberry 2004:10-11; J. Martin 2000:60)
      % General make-up today
        % The increase in the Latinx-dominant neighborhoods in the US has mostly occurred in the western states (i.e., not Louisiana), according to census records between 1990 and 2010 (Wright et al. 2014:179)
    \subsection{Creole identity}
      % It has been suggested that 4 factors have helped maintain an interest in Creole identity up to the present: 1) A concern that Louisiana exports being marketed as "Cajun" will erase Creole contributions, 2) the lack of laws or even census options that would pigeonhole Creoles into identifying as Black instead, 3) the focus on French language education, and 4) the popularity of geneological research (Susberry 2004:14)
      % For Creoles, Susberry (2004) divides their historical development as a group into 4 stages: 1) dissonance, 2) struggle for validation, 3) negotiation of group identity, and 4) activism (8)
        % At each stage, factors influencing Creoles' self-conceptualizations included 1) physical appearance, 2) low social status of Blacks, 3) negative treatment by Whites and Blacks, and 4) members' consciousness of the group's sociopolitical evolution (Susberry 2004:8)
      % Around the time of the Louisiana Purchase, Creoles were White descendants of Europeans born in the Americas (Kein 2000, as cited in Susberry 2004:7-8)
      % After the Louisiana Purchase, Creole eventually came to be the name for "mixed race people of color" (Martin 2000, as cited in Susberry 2004:7-8)
      % White Americans and White Creoles were already attempting to socially collapse Free People of Color in with enslaved Blacks by 1830, though these efforts weren't very successful (Daniel 1992; Gehman 1994; Bryan 2000, all as cited in Susberry 2004:10-11)
      % After the Civil War, poverty (Gehman 1994) and the abolishment of slavery became factors that made it easier for Whites to collapse Creoles of Color into the same social category as Blacks (previously enslaved people) (Susberry 2004:11)
        % In the 20th century, Creoles of Color themselves tended to align themselves with Blacks between the Civil War and the Civil Rights Movement despite seeing themselves as culturally distinct (Dubois & Melançon 2000; Gehman 1994, both as cited in Susberry 2004:12)
        % For Creoles of Color who resented being classified as Black would either try to pass as White if possible or simply assert that they were racially Creole rather than Black (Susberry 2004:12-13)
      % Some definitions used by researchers today
        % "Multiracial people whose ancestral history is connected to Louisiana" (Susberry 2004:8)
        % Mixed race people with French cultural origins (Giancarlo 2019:34), though she argues that a common thread in thread in researchers' definitions is that Creoles are necessarily not Cajuns
    \subsection{Cajun identity}
      % Cajuns were historically seen as racially "ambiguous", "not black, but seen as less white than the Anglo-Saxons" (Tentchoff 1980; Walton 2003, both as cited in Giancarlo 2019:32)
        % Indeed, Cajuns began owning enslaved people in the 1770s and most did by 1810 (Brasseaux 1985), most being as much as 68% (Baker 1974, both as cited in Giancarlo 2019:31)
      % Cajun membership
        % Other ethnic groups (Scots-Irish, Italians, Germans, some Native Americans) have been incorporated in the Cajun ethnicity, so ancestry and last names have been argued to not be necessary conditions for Cajun identity (Brasseaux 1992; Stanford 2016, both as cited in Giancarlo 2019:33)
      % Giancarlo (2019) argues that characterizing all White South Louisianians as Cajun "minimizes a diversity of white ethnicities" (33)
        % She seems to mean that White South Louisianians might otherwise maintain different ethnic identities rather than that the racial identities aligned with Cajun identity
    \subsection{Language ideologies accounting for race or ethnicity}
      % Study participants in St Landry claimed to speak French if they identified as Cajun and Creole if they identified as Creole, regardless of whether their speech appeared structurally more like French or more like Creole (Klingler 2003 "Language")
      % Louisiana isn't the only place where race and ethnicity influence language ideologies.
        % During the Quiet Revolution in Quebec, one might hurl an insult at people speaking French in public by demanding that they "speak White" (i.e., English) instead (Lamarre 2014:149)
  \section{French and Creole}
    % Glossonym
      % Creole speakers in Pointe Coupee (determined by the language structure) mostly call their language "créole" but sometimes also "français" or "cadien", though even when using the latter two glossonyms, speakers note that their language isn't like French from France or Cajun from Lafayette (Klingler 2003 "Turn":128)
    % Where is situated socially and geographically?
    \subsection{Status in Louisiana}
      % Where did French come from in the first place
        % Although Acadian French is at least implicitly considered the main origin of French in rural Louisiana, there are many sources for French (Klingler 2009)
        % Gudmestad & Carmichael (2022) had a number of study participants (Indians) who were native speakers in their 30s (in 2007-2008) in Terrebonne-Lafourche (6-7)
      % Where did Creole come from in the first place
        % People who were enslaved and transported from Senagambia mainly spoke Malinke/Maninka, Serer, Wolof, or Pulaar (Klingler 2003 "Turn":57)
        % African languages may be responsible features such as (Klingler 2003 "Turn":62-66):
          % Predicate clefting in which an adjective is repeated in its usual location (e.g., Se malad li malad)
          % Preverbal markers
          % Frequent copula absence
          % Postnominal definite and demonstrative determiners
          % The plural postnominal definite article being identical to the 3pl subject pronoun
    \subsection{French vs Creole}
      % French and French-related varieties that exist
        % "Cajun French" (though he prefers "Louisiana Regional French" as it is not spoken purely by Cajuns), "Louisiana Creole", and "Colonial French" (which he and Picone (1998) prefer to call Plantation Society French since it developed in the 19th century) (Klingler 2003 "Language":77)
      % Speakers' approaches to naming their languages as one or the other
        % It's not unusual to place languages under a label based on cultural traits of the speakers rather than the structure of their ways of speaking
          % Linguists once described Cangin as a variety of Serer because Wolofs in the area saw the cultural practices of both Cangin and Sereer speakers as being the same (Irvine & Gal 2000:57)
        % When asked plainly (without pushing for greater specificity), speakers of both varieties in St Landry will call their language "French" (Klingler 2003 "Language":78-79)
          % Apparently Spitzer (1977) and Le Menestrel (1999) have also noted this labeling approach by speakers.
      % The issue of maintaining two discrete social constructs
        % This has been an issue for AAL
          % Baugh (2015) was emphatic that AAVE is English (766-767)
      % Structural blurring of boundaries between French and Creole
        % This is particularly prevalent in Lafayette, Breaux Bridge, and Vacherie (Klingler 2003 "Language":78), the former two being where much of my data comes from.
        % Klingler (2003 "Language") uses 1sg subject pronouns, past perfective constructions, and the verb 'to have' as diagnostics for structurally disentangling French and Creole (80)
          % 1sg was a better option than other pronouns, such as 3sg, because in the latter case, the pronouns il and li can be reduced in both French and Creole, respectively, to [i] (Klingler 2003 "Language":80)
            % This is notable because it inadvertantly gives us another example of the difficulty in drawing a structural boundary between French and Creole
      % Spelling convention used in this study
        % It has been argued for British Creole that non-standard spelling functions both as a way to represent a language that has no orthographic standard as well as a way to distance itself from its superstrate when words could be spelled the same but aren't, such as <Jameka> or <kool> (Sebba 1998, as cited in Androutsopoulos 2000:515)
        % Bengali and Hindi speakers who mix English into their online writing in the former languages sometimes write the former languages using the Roman script for convenience (Barman et al. 2014:13)
          % Likewise, writing here using French orthographic conventions is not a stance on the primacy of French but rather a matter of convenience
    \subsection{Distinguishing features}
      % Geographic variation
        % Almost all of the variants marked as Creole (by Klingler, 1sg subject pronouns, perfective aspect representation, and the verb 'to have') used by speakers in St Landry Parish were produced by 4 speakers from Leonville, Prairie Ville, and Arnaudville (Klingler 2003 "Language":83).
          % These towns are all along Bayou Teche which historically was lined with plantations (Klingler 2003 "Language":83)
          % Most variants were marked as French in general
          % Most Creole variants were also used by Blacks, though not exclusively (some White Cajuns used Creole-marked perfective aspect)
        % Almost all variants of 1sg pronouns, perfective aspect, and the verb 'to have' used by speakers in Pointe Coupee (3 Black, 2 White) were variants marked as Creole save for some tokens (~7%) of 'avoir' used only by Whites (Klingler 2003 "Language":81)
          % For 1sg specifically, only 2 tokens were not mo, and those 2 tokens were null rather than je (Klingler 2003 "Language":81)
    \subsection{Subject pronoun system}
      % Terrebonne-Lafourche
        % For 1sg, null was the most common for Indian speakers of all levels, sometimes with moi (i.e., moi by itself was possible) (Gudmestad & Carmichael 2022:11)
      % Not Terrebonne-Lafourche
    \subsection{Ethnically conditioned pronouns} % Add general social and structural constraints to pronoun predictors chapter
      % 1sg as [z] in Terrebonne-Lafourche may be indicative of Indian identity (Dajko 2009, as cited in Gudmestad & Carmichael 2022:5)