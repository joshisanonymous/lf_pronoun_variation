\chapter{Methods}
  \section{The location}
    \subsection{Previous descriptive work on the area}
    \subsection{My ethnographic work on the area}
  \section{Sampling approach and data collection}
    % Finding Creoles
      % Susberry (2004) used email lists, organizations, Creole ancestry conferences, Catholic churches, and universities to find participants (34)
        % For myself, the most effective approach was to visit small, local businesses and organizations and speak to customers and employees
        % Social media (reddit) and attending local Creole-centric events helped, as well, whereas churches were not responsive
          % Likewise, Susberry (2004) did not always have success with contacting churches but rather had most of her success with online surveys (51)
    % Frequency of each pronoun
      % 1st and 2nd person could be expected to be more frequent in speech (and CMC) than they would be in writing (Yates 1996:40-41, also citing Chafe 1982 and Fowler & Kress 1979)
  \section{Data coding}
    % Not only were participants anonymized but also non-participants and even famous or historical figures
      % Ethical data collection can be an issue when collecting information about non-participants as these people are not providing consent (Stark 2018:244)
    \subsection{Participants}
      % Racial coding
        % Several general approaches that include biracial people
          % The Five Identities for Mixed-Race People scheme includes 1) foreclosed identity (accepting without question a minority identity assigned by society), 2) single [minority] identity (chosen by the individual), 3) mixed identity (e.g., something hyphenated), 4) new race identity (e.g., "Biracial" or perhaps "Creole"), and 5) white identity (chosen by the individual) (Root 2003, as cited in Susberry 2004:23)
        % Susberry (2004) classified Creoles of Color following Rockquemore (1999) as Black, protean (shifting), or border (biracial) (33)
      % Socioeconomic class
        % Some indicators of membership in the middle-class are income based (Massey & Fischer 1999; Adelman 2004; Haynes 2001; Lacy 2007) while others are occupation based (Landry 1987, all as cited in Britt & Weldon 2015:3)
        % Trudgill (1974) argued that occupation is the primary indicator for socioeconomic class because even affluent manual laborers would likely retain the culture of manual laborers (as cited in Dodsworth 2011:193)
        % Based essentially on occupation type, dividing people in Estate Class (fieldworkers) and Non-Estate Class (foremen and those unassociated with the estates) was meaningful socially and in terms of language variation in Guyana (Rickford 1986:216-217)
        % In Philadelphia, occupation was a better predictor of socioeconomic class than other indicators, but a combination of indicators was the best predictor (Labov 2001,as cited in Dodsworth 2011:193)
        % Employment was the best indicator of presitge in Cusco, Peru (Van den Berghe & Primov 1977, as cited in Delforge 2012:321)
        % As has long been noted, it's simply easier to obtain occupational information than other indicators of socioeconomic class Macaulay (1977, as cited in Dodsworth 2011:193)
        % One's speech patterns can impact their ability to get into certain occupational positions
          % In the US, this has been true for Chinese, Indian, and Mexican accents (Timming 2017:416-422) as well as Japanese but not [probably standard] French accents (Hosoda & Stone-Romero 2010, as cited in Timming 2017:412)
          % Jerry Pierre Quintero joked about a similar difficulty with his speech in English when he first looked for work
    \subsection{Pronouns}
