\chapter{Methods}
  To respond to the research questions above, variationist study design will be used, meaning participants will be recruited, interviews will be conducted and recorded, and the data will be analyzed statistically and textually.
  Research questions 1 and 2 are quantitative in nature and so will be given statistical treatments, whereas research question 3 will be given a qualitative treatment.
  The aim is to get a broad picture of how ethnicity interacts with Louisiana French subject pronoun usage and what those interactions might mean about the relationship between Creoles and Cajuns.

  \section{Lafayette and the surrounding parishes}
    The data that will be used for this study will come primarily from the parishes surrounding Lafayette Parish in the heart of South Louisiana as well as from Lafayette Parish itself.
    These parishes have been chosen for the high likelihood that both people identifying as Cajuns and Creoles will be found there.
    As practically all of South Louisiana has been branded Cajun \parencite{giancarlo_dont_2019}, members of this ethnic group can be found most anywhere.
    However, those identifying as Creoles are not found everywhere.
    Both \textcite{rottet_language_1995} and \textcite{dajko_ethnic_2009}, collecting data from Southeast Louisiana, relied on Cajuns and Houma Indians for ethnic categories as they did not find Creoles who spoke French in this region.
    On the other hand, the most comprehensive descriptions of Louisiana Creole came from data gathered in Pointe Coupee Parish \parencite{klingler_if_2003} and St Martin Parish \parencite{neumann_creole_1985}, the latter of which is adjacent to Lafayette Parish.
    Given the tendency of speakers to conflate their ethnic identities with what they call their language varieties (\citeauthor{brown_pronominal_1988}, \citeyear[p.~5]{brown_pronominal_1988}; \citeauthor{klingler_language_2003}, \citeyear{klingler_language_2003}), it is reasonable to assume this to mean that there are also self-identifying Creoles residing in these areas.
    Other parishes where it has been noted that Louisiana Creole is spoken include St James Parish and Cameron Parish \parencite[p.~7]{rottet_language_1995}.
    Additionally, descriptive linguistic work has been done in Opelousas in St Landry Parish that specifically relied on finding French speakers who identified as Creoles \parencite{klingler_language_2003}.
    Of the parishes mentioned, St Landry Parish and St Martin Parish are directly adjacent to Lafayette Parish while the rest are more distant with Pointe Coupee Parish being the next closest as it is adjacent to both St Landry and St Martin.

  \section{Sampling}
    In previous work in rural South Louisiana, snowball sampling has proven effective \parencite{brown_pronominal_1988, giancarlo_dont_2019, rottet_language_1995}, so this approach will be used in the present study, as well, albeit taking several different starting points.
    One advantage of this approach is that, while personal social networks rather than entire community networks are of interest in the present study, it could still be advantageous to see some overlap in personal networks.
    On the other hand, using different starting points will lend itself to approaching a balanced sample though likely not a true representative sample.
    Outside of speaking French -- or Creole if they identify their language that way -- the only characteristic participants need have in order to be included is that they identify as either Cajun or Creole, regardless of any other identifications they might have, which will be used as a factor in statistical analyses.
    Speakers who did not learn French at home will still be included in the study as it has previously been noted that observing the differences between native speakers and ``New Speakers'' in the context of language revitalization could be enlightening \parencite[p.~21]{gudmestad_variationist_2022}.
    Language background will thus be included as a social variable, specifically whether French was learned in a naturalistic settings, such as the home, or an institutional setting or through personal study.

  \section{Social variables}
    Other life history information about participants that will be collected will include birth year, residence, where they were raised, profession, education, and race.
    In previous studies, age was shown to be an important factor in subject pronoun variation \parencite{rottet_language_1995}.
    Profession and education were not included in previous work due to reports from participants that socioeconomic differences were a relatively recent phenomenon in rural Louisiana (\citeauthor{dajko_ethnic_2009}, \citeyear[p.~66]{dajko_ethnic_2009}; \citeauthor{rottet_language_1995}, \citeyear[p.~64]{rottet_language_1995}), though they will be included in the present study as more time has passed, and the Lafayette area, being the third largest metropolitan area in Louisiana, likely began experiencing some socioeconomic stratification earlier than Terrebonne and Lafourches Parishes, where the cited works took place.

    The levels for the social variables are summaried in Table \ref{tab:social_vars}, though some notes are necessary.
    Firstly, residence and where participants were raised are expected to aggregate best by parish, though if there is a confluence of individual towns, the levels will be adapted to go by town instead.
    Secondly, for profession and education, it would be ideal to operationalize these using the linguistic market \parencite{sankoff_linguistic_1978}, though this would require spending more time during interviews discussing professional and educational backgrounds than is desired as the true focus of this study is ethnicity and race.
    The collar system is a compromise as it is still used in sociolinguistics today \parencite[e.g.,][]{dodsworth_social_2017, forrest_community_2015}.
    For those who are retired, which will likely be many of the participants, they will be categorized according to what their profession was before they retired.
    As for education, if it is found that a number of participants have graduate degrees, a fourth level will be added to this variable.

    \begin{table}[tbhp]
      \centering
      \caption{Summary of social variables}
      \label{tab:social_vars}
      \begin{tabular}{l l}
                          & \\
        Social Variable   & Levels \\
        \hline
        Ethnicity         & Creole, Cajun \\
        French Background & naturalistic, institutional, personal \\
        Gender            & man, woman, other answer \\
        Birth Year        & continuous numeric \\
        Residence         & parish \\
        Raised            & parish \\
        Profession        & blue and white collar \\
        Education         & some school, high school graduate, college graduate \\
        Race              & singular White, singular Black, border, protean, transcendent
      \end{tabular}
    \end{table}

    Finally, unlike ethnicity which will be presented as a binary choice for participants, race will involve self-identification through an open-ended question as open-ended questions have been found to produce more accurate reporting than closed-ended questions \parencite[p.~434]{vannette_proxy_2018}.
    As has been suggested in sociolinguistic research on biracial people \parencite{holliday_multiracial_2019} and has been shown for Creoles in particular \parencite{susberry_racial_2004}, there are many ways that mixed-race people may choose to racially identify.
    Ultimately, race will be coded using \citeauthor{rockquemore_beyond_2007}'s (\citeyear{rockquemore_beyond_2007}) typology wherein one may have a singular racial identity (in this case, likely Black or White), a border racial identity (i.e., consistently identifying as mixed, in this case, likely Creole), a protean racial identity (i.e., changing relative to the situation), or a transcendent racial identity (i.e., rejecting race as a concept altogether).

    Outside of the life histories of participants, the ethnic and racial characteristics of those they name as part of their personal networks (i.e., who are called the participants' alters) will be obtained.
    This information will be gathered from the participants themselves rather than from those alters directly.
    While this potentially yields categories that do not match what those alters would have given, this is not expected to be problematic.
    As has been noted in the literature \parencite[pp.~245-246]{vannette_collecting_2018}, the participants' perception of the characteristics of those in their networks is often more meaningful than their true characteristics, and that is likely to hold for the present study.
    Furthermore, surveys participants' reports on relatively fixed details of non-participants who are close to them have proven to be nearly as accurate as what would have been reported by the non-participants themselves \parencite{vannette_proxy_2018}.

  \section{Linguistic variables}
    Table \ref{tab:lf_sub_pro} outlines the 11 dependent linguistic variables that will be analyzed in this study.
    Some of these are, however, expected to collapse and are included here more for completeness and to avoid assumptions.
    To start, gender in the 3rd person singular inanimate pronouns is not likely to be overtly expressed as \lexi{elle} and \lexi{il}, the only two variants that are particular to one gender or the other, are not commonly used for inanimate referents in French outside of writing, and most Louisiana French speakers are not literate in French.
    Likewise, gender in 3rd person plural pronouns is not often expressed in any variety of French as the feminine form \lexi{elles} is reserved for groups in which the referents are all feminine, whereas \lexi{ils} is used if any referent at all is masculine.
    This is also the case in Louisiana French, which goes further in providing several other variants for 3rd person plural that are unmarked for gender.
    Table \ref{tab:lf_pro_collapse} therefore summarizes the linguistic variables that are expected to collapse in the data.

    \begin{table}[tbhp]
      \centering
      \caption{Linguistic variables that are expected to collapse into single variables}
      \label{tab:lf_pro_collapse}
      \begin{tabular}{l @{ + } l @{ $\to$ } l}
        \multicolumn{3}{c}{} \\
        (3sg.IF) & (3sg.IM) & (3sg.I) \\
        (3pl.F)  & (3pl.M)  & (3pl)
      \end{tabular}
    \end{table}

    Structural independent variables will also be included, namely the type of verb following the pronoun, be it a lexical verb, modal verb, or auxiliary verb.
    Personal experience and a small pilot study revealed that the use of \lexi{ils} for the 3rd person plural is heavily favored before the auxiliary verbs \lexi{être} \gloss{to be} and \lexi{avoir} \gloss{to have} even when it is disfavored for all other verbs types.

  \section{Data collection}
    % interview module stuff
    All life history, social network, and linguistic data will be collected through semi-structured sociolinguistic interviews.
    Interviews will be conducted one-on-one by myself in Louisiana French with the goal of collecting at least 30 hours of speech, ideally collecting 45-60 minutes from each speaker.
    While my own subject pronoun patterns could influence the pronouns used by participants, I will take care to be consistent from interview to interview so as to control for this effect to the extent that it is possible to do so.

    Interviews will begin with collecting life history information other than race followed by casual conversation until discussing ethnicity and race in the later portions of the interviews and finally questions about social networks alters.
    In particular, as a wide-range of subject pronouns are desired, the casual conversation section will lean towards asking participants to relate stories that might involve describing events from different points of view.

    Social network questions will follow the typical practice of using name-generator questions to obtain alters followed by name-interpreter questions to obtain details of those alters.
    While using computer technology to ask these questions likely reduces the mental burden placed on participants \parencite[Stark \& Krosnick, 2017, as cited in][pp.~248-249]{vannette_collecting_2018}, these questions will instead be asked orally as many of the participants will likely be advanced in age and perhaps uncomfortable with computers.
    Additionally, asking orally allows the questions to be posed in French as most Louisiana French speakers do not read and write in French.
    This is beneficial as it keeps them in the French mode during the interview and will possibly cause them to report more of the francophones to whom they are close than they might otherwise do.
    Additionally, when posing name-interpreter questions, the participants will be asked to give one type of characteristic for all their alters before moving on to the next type of characteristic as this is known to yield data with more validity \parencite[Coromina \& Coenders, 2006; Vehovar et al., 2008, both as cited in][p.~247]{vannette_collecting_2018}.
    The exact name-generator and name-interpreter questions to be used, as well as the rest of the interview module, are included in Appendix \ref{app:module}.

  \section{Analyses}
    As this study is mostly quantitative in nature, fairly standard variationist methodology will be used in the analysis.
    Each person-number possibility for subject pronouns, as summarized in Table \ref{tab:lf_sub_pro}, will act as a linguistic variable.
    Where these linguistic variables have only two categorical variants, binomial logistic mixed-effects models will be fit, whereas where there are more than two variants, multinomial logistic mixed-effects models will be fit as was done in \textcite{gudmestad_variationist_2022} for 1st person singular subject pronouns in Louisiana French.
    The exact formula to be used for analyses, given in R formula notation, is as shown in Equation \ref{eq:model}.
    In this formula, Pronoun is the dependent linguistic variable, each additional term is a factor, and (1|Participant) as well as (1|Following.Verb) are a random intercepts.
    In particular, the following verb is a random intercept despite the use of verb type as a factor as some lexical verbs will likely be overrepresented, such as \lexi{savoir} \gloss{to know} and \lexi{penser} \gloss{to think}.
    The R function used to fit the binomial models will be \code{glmer} from the \code{lme4} package \parencite{bates_lme4:_2019} and to the function for the multinomial models will be \code{mblogit} from the \code{mclogit} package \parencite{elff_mclogit_2022}.

    \begin{equation}
      \label{eq:model}
      \begin{aligned}
        \text{Pronoun}\  & \~{}\  \text{Verb.Type}\  +\  \text{French.Background}\  +\  \text{Birth.Year}\  +\  \text{Gender}\  +\\
                         & \text{Ethnicity}\  +\  \text{Race}\  +\  \text{Raised}\  +\  \text{Residence}\  +\  \text{Profession}\  +\  \text{Education}\  +\\
                         & \text{Network.Ethnic.Homophily}\  +\  (1|\text{Participant}) + (1|\text{Following.Verb})
      \end{aligned}
    \end{equation}

    In general, AIC will be used for model selection, though an additional consideration will be examined before calculating the AIC of different models.
    While both ethnicity and race are included in the model formula, it is quite possible that there will be a strong association between the two.
    As such, the VIF will be calculated for both factors as well as all other factors to determine if any are highly collinear, in that their VIFs are 5 or higher, in which case those factors deemed less central to the study's goals will be removed from the relevant models.
    At this point, AIC will be taken into consideration to find a good model fit.
    Finally, $P$-values for factors will only be considered once model selection has been completed.

    Included in the above model is Network Ethnic Homophily, which represents the degree to which the alters in participants' personal networks have the same ethnicity as the participants themselves and is calculated using the EI homophily index \parencite{lizardo_social_2020} shown in Equation \ref{eq:ei_homoph} where External refers to those unlike ego and Internal to those like ego.
    While this is technically part of the second research question examining social networks, it fits best as a factor in the original models.
    Additional analyses will be carried out to better understand social networks beyond what the models above show.

    \begin{equation}
      \label{eq:ei_homoph}
      \text{EI Homophily} = \frac{\text{External} - \text{Internal}}{\text{External} + \text{Internal}}
    \end{equation}

    Important to note for the name-generator questions used here is that participants are not limited to naming alters with whom they speak French.
    Indeed, the first name-generator question targets alters who are particularly close to the participant whom one might call core alters \parencite{marsden_core_1987}.
    As such, a multinomial logistic mixed-effects model will be fit to better understand the relationship between coreness and language used, where French usage with the alter will be categorized as never occurred, occurring occasionally, occurring frequently, or always occuring, and alter types will be either core or non-core.
    The formula used can be seen in Equation \ref{eq:core_model} where participant is again a random intercept.

    \begin{equation}
      \label{eq:core_model}
      \text{French.Frequency}\  \~{}\  \text{Alter.Type}\  +\  (1|\text{Participant})
    \end{equation}

    To add to the relationship between coreness and French usage, narrower comparisons of network homophily will also be tested, as summarized in Table \ref{tab:homophily}.
    Specifically, a paired $t$-test will be performed for the difference in means between the ethnic homophily of the non-francophone alters in participants' personal networks and the ethnic homophily of the francophone alters, which will show whether French is used more or less with ethnically similar alters.
    Another $t$-test for the difference in means, non-paired, will also be performed between the overall ethnic homophily of Cajun participants and overall ethnic homophily of Creole participants to identify whether the groups differ in the ethnic diversity of their networks.

    \begin{table}[tbhp]
      \centering
      \caption{Homophily hypothesis tests}
      \label{tab:homophily}
      \begin{tabular}{l l l l}
                                                    &                                                 &                 & \\
        Stat                                        & Stat                                            & Test            & $N$ \\
        \hline
        \pbox{\textwidth}{Mean ethnic homophily\\of francophone alters} & \pbox{\textwidth}{Mean ethnic homophily\\of non-francophone alters} & Paired $t$-test & 60 \\
        \hline
        \pbox{\textwidth}{Mean ethnic homophily\\for Creoles}           & \pbox{\textwidth}{Mean ethnic homophily\\for Cajuns}             & $t$-test        & 30 \\
      \end{tabular}
    \end{table}
