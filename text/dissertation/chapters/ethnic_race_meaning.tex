\chapter{Social Meaning of Ethnicity and Race in Louisiana}
  % Justify this chapter
  \section{Previous work on how Louisianians conceptualize ethnicity and race}
    \subsection{Creoles}
      % Recent definitions of Creole identity by Creoles themselves are often racialized
        % Creoles of Color highlighted mixed ancestry that includes French and Spanish and being from specifically south Louisiana (Susberry 2004:56)
          % Thought some of Susberry's (2004) participants thought of Creole as a cultural concept, too (56)
      % Stances towards one's own Creole ethnic identity
        % Some of Suberry's (2004) Creole of Color participants did not find out about their Creoles ancestry until later in life but then embraced this as their ethnicity (59)
        % Tajfel has argued that social identity helps one maintain self-esteem, and Phinney (1992) has argued the same for ethnic minorities' commitment to their own ethnic identities (as cited in Susberry 2004:21-22)
          % Susberry (2004) did not find an association between ethnic identity achievement and self esteem for her own Creole of Color participants (50)
      % How Creoles have racially self-identified
        % Study participants in St Landry identifying as Creole (N = 19) all also identified as Black (Klingler 2003 "Language":82)
        % Susberry (2004) classified Creoles of Color following Rockquemore (1999) as Black, protean (shifting), or border (biracial) (33)
          % Susberry's (2004) 85 Creoles of Color identified as 41.2% singular Black, 42.4% protean, and 16.5% border (45)
            % There were additionally 9 transcendent Creoles of Color in the data (Susberry 2004:35)
      % Factors in the racial self-identifications of ethnic Creoles
        % 8 factors are included in the Factor Model of Multiracial Identity: 1) ancestry, 2) early socialization experiences, 3) cultural attachment, 4) appearance, 5) social and historical context of racial groups, 6) political awareness and orientation, 7) other social identities (e.g., gender), and 8) spirituality (Wijeyesinghe 2001, as cited in Susberry 2004:24-25)
        % In general, Susberry (2004) found that ethnic identity achievement, self-esteem, self-perceived skin color, and "Creole dialect" fluency had statistically significant impacts on racial identifications, but negative treatment by Whites, negative treatment by Blacks, and the perceived social status of Blacks were not statistically significant (47)
        % Creoles of Color who identified as Black believed they appeared Black to others whereas those who identified as Creole believed they appeared ambiguous to others, suggesting a link between beliefs about their own appearances to others and how they choose to self-identify (Susberry 2004:45/48)
          % An implication is that one needs to be able to at least believe that society doesn't see them as Black in order for them to self-identify as Creole, though it's not clear if this held true for my own participants.
        % The racial make-up of Creoles of Color's childhood friends had an impact on how they racially self-identified (Susberry 2004:45/48):
          % Singular Blacks had mostly Black friends
            % This makes sense assuming that their friends were assertive about their Black identities.
          % Border Creole had mostly White friends
            % This feels like an acknowledgement that one isn't or can't be seen as White while not wanting to distance oneself from friends by claiming a singular Black identity
          % Protean had an even mix of friends of various racial identities
    \subsection{Cajuns}
      % Cajun Renaissance
        % CODOFIL only became "pro-Cajun" in the 1990s (Sexton 1999, as cited in Giancarlo 2019:32)
      % How Cajuns have racially self-identified
        % Study participants in St Landry identifying as Cajun (N = 9) all also identified as White (Klingler 2003 "Language":82)
    \subsection{Conflict between Creoles and Cajuns}
      % Stanford (2016) [through personal anecdote?] suggests that Louisianians will quickly explain that Creoles are Black and Cajuns are White even if they admit that there are lighter skinned Creoles and darker skinned Cajuns (as cited in Giancarlo 2019:32)
      % In Lafayette and St Landry Parishes, basing identity on ancestry and basing ancestry on family names, Creoles are the dominate group (Giancarlo 2019:33)
      % Erasure of Creole contributions to South Louisiana culture
        % When the Cajundome in Lafayette was named, public arguments arose with, for example, one Black activist, Takuna Maulana El Shabazz (1992), writing that this was a form of racist colonization (1992), and one prominent Cajun folklorist, Barry Ancelet (1992), describing said criticism as a case of reverse racism (as cited in Giancarlo 2019:36)
        % Louisiana food is often marketed as Cajun, but study participants and others (Herbert Wiltz, Dormon 1983) have argued that the food is actually Creole as it was enslaved people doing the cooking when these dishes were developed (Giancarlo 2019:39-40)
  \section{Analysis}
    \subsection{Themes 1, 2, ...}

  % Worldwide, the increase in income inequality has been "unprecedented among the world's richest democracies" at the end of the 20th century (Neckerman 2004, as cited in J. A. Smith et al. 2014:436)