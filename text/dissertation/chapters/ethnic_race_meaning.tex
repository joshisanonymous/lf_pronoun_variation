\chapter{Social Meaning of Ethnicity and Race in Louisiana}
  % Justify this chapter
  \section{Background}
    \subsection{What do ethnicity and race mean in the US}
    \subsection{Previous work on how Louisianians conceptualize ethnicity and race}
      \subsubsection{Creoles}
        % Stances towards one's own Creole ethnic identity
          % Some of Suberry's (2004) Creole of Color participants did not find out about their Creoles ancestry until later in life but then embraced this as their ethnicity (59)
        % Factors in the racial self-identifications of ethnic Creoles
          % 8 factors are included in the Factor Model of Multiracial Identity: 1) ancestry, 2) early socialization experiences, 3) cultural attachment, 4) appearance, 5) social and historical context of racial groups, 6) political awareness and orientation, 7) other social identities (e.g., gender), and 8) spirituality (Wijeyesinghe 2001, as cited in Susberry 2004:24-25)
          % Creoles of Color who identified as Black believed they appeared Black to others whereas those who identified as Creole believed they appeared ambiguous to others, suggesting a link between beliefs about their own appearances to others and how they choose to self-identify (Susberry 2004:45/48)
            % An implication is that one needs to be able to at least believe that society doesn't see them as Black in order for them to self-identify as Creole, though it's not clear if this held true for my own participants.
          % The racial make-up of Creoles of Color's childhood friends had an impact on how they racially self-identified (Susberry 2004:45/48):
            % Singular Blacks had mostly Black friends
              % This makes sense assuming that their friends were assertive about their Black identities.
            % Border Creole had mostly White friends
              % This feels like an acknowledgement that one isn't or can't be seen as White while not wanting to distance oneself from friends by claiming a singular Black identity
            % Protean had an even mix of friends of various racial identities
  \section{Analysis}
    \subsection{Themes 1, 2, ...}

  % Worldwide, the increase in income inequality has been "unprecedented among the world's richest democracies" at the end of the 20th century (Neckerman 2004, as cited in J. A. Smith et al. 2014:436)