%%%%%%%%%%%%%%%%%%%%%%%%%%%%%%%%%%%%%%%%%%%%%%%%%%%%%%%%%%%%%%%%
% Joshua McNeill                                               %
% joshua dot mcneill at uga dot edu                            %
%%%%%%%%%%%%%%%%%%%%%%%%%%%%%%%%%%%%%%%%%%%%%%%%%%%%%%%%%%%%%%%%

\documentclass{beamer}\usepackage[]{graphicx}\usepackage[]{xcolor}
% maxwidth is the original width if it is less than linewidth
% otherwise use linewidth (to make sure the graphics do not exceed the margin)
\makeatletter
\def\maxwidth{ %
  \ifdim\Gin@nat@width>\linewidth
    \linewidth
  \else
    \Gin@nat@width
  \fi
}
\makeatother

\definecolor{fgcolor}{rgb}{0.345, 0.345, 0.345}
\newcommand{\hlnum}[1]{\textcolor[rgb]{0.686,0.059,0.569}{#1}}%
\newcommand{\hlsng}[1]{\textcolor[rgb]{0.192,0.494,0.8}{#1}}%
\newcommand{\hlcom}[1]{\textcolor[rgb]{0.678,0.584,0.686}{\textit{#1}}}%
\newcommand{\hlopt}[1]{\textcolor[rgb]{0,0,0}{#1}}%
\newcommand{\hldef}[1]{\textcolor[rgb]{0.345,0.345,0.345}{#1}}%
\newcommand{\hlkwa}[1]{\textcolor[rgb]{0.161,0.373,0.58}{\textbf{#1}}}%
\newcommand{\hlkwb}[1]{\textcolor[rgb]{0.69,0.353,0.396}{#1}}%
\newcommand{\hlkwc}[1]{\textcolor[rgb]{0.333,0.667,0.333}{#1}}%
\newcommand{\hlkwd}[1]{\textcolor[rgb]{0.737,0.353,0.396}{\textbf{#1}}}%
\let\hlipl\hlkwb

\usepackage{framed}
\makeatletter
\newenvironment{kframe}{%
 \def\at@end@of@kframe{}%
 \ifinner\ifhmode%
  \def\at@end@of@kframe{\end{minipage}}%
  \begin{minipage}{\columnwidth}%
 \fi\fi%
 \def\FrameCommand##1{\hskip\@totalleftmargin \hskip-\fboxsep
 \colorbox{shadecolor}{##1}\hskip-\fboxsep
     % There is no \\@totalrightmargin, so:
     \hskip-\linewidth \hskip-\@totalleftmargin \hskip\columnwidth}%
 \MakeFramed {\advance\hsize-\width
   \@totalleftmargin\z@ \linewidth\hsize
   \@setminipage}}%
 {\par\unskip\endMakeFramed%
 \at@end@of@kframe}
\makeatother

\definecolor{shadecolor}{rgb}{.97, .97, .97}
\definecolor{messagecolor}{rgb}{0, 0, 0}
\definecolor{warningcolor}{rgb}{1, 0, 1}
\definecolor{errorcolor}{rgb}{1, 0, 0}
\newenvironment{knitrout}{}{} % an empty environment to be redefined in TeX

\usepackage{alltt}
  % Beamer settings
  \usetheme{CambridgeUS}
  \usecolortheme{seagull}
  \usefonttheme{professionalfonts}
  \usefonttheme{serif}
  \setbeamertemplate{bibliography item}{}

  % Packages and settings
  \usepackage{fontspec}
    \setmainfont{Charis SIL}
  \usepackage[backend=biber, style=apa]{biblatex}
    \addbibresource{../References.bib}
  \usepackage{hyperref}
    \hypersetup{colorlinks=false}
  \usepackage{graphicx}
    \graphicspath{{../../data/photos/}{./present_figures/}}
  \usepackage[normalem]{ulem}

  % Document information
  \author{Joshua McNeill}
  \title[Divisions raciales dans les pronoms]{L'expression des divisions raciales dans les pronoms sujets de la troisième personne au pluriel en français louisianais}
  \institute[UGA]{
    \url{joshua.mcneill@uga.edu} \\
    \vspace{0.5cm}
    University of Georgia}
  \date{26 septembre 2024}
  \titlegraphic{\includegraphics[scale=0.038]{uga_logo.png}}
  \newcommand{\Logo}{{\hskip0pt plus 1filll \includegraphics[scale=0.028]{uga_logo.png}}}

  %% Custom commands
  % Lexical items
  \newcommand{\lexi}[1]{\textit{#1}}
  % Gloss
  \newcommand{\gloss}[1]{`#1'}
  \newcommand{\tinygloss}[1]{{\tiny`#1'}}
  % Orthographic representations
  \newcommand{\orth}[1]{$\langle$#1$\rangle$}
  % Utterances (pragmatics)
  \newcommand{\uttr}[1]{`#1'}
  % Sentences (pragmatics)
  \newcommand{\sent}[1]{\textit{#1}}
  % Smaller citations
  \newcommand{\smallcite}[1]{{\scriptsize{}#1}}
  % Research questions
  \newcommand{\RQone}{What factors lead to more or less use of French in the linguistic landscape?}
\IfFileExists{upquote.sty}{\usepackage{upquote}}{}
\begin{document}
\begin{knitrout}
\definecolor{shadecolor}{rgb}{0.969, 0.969, 0.969}\color{fgcolor}\begin{kframe}


{\ttfamily\noindent\color{warningcolor}{\#\# Warning: Inner iterations did not coverge - nlminb message: false convergence (8)}}\begin{verbatim}
## 
## Iteration 1 - deviance = 862.8423 - criterion = 0.6396059
\end{verbatim}


{\ttfamily\noindent\color{warningcolor}{\#\# Warning: Inner iterations did not coverge - nlminb message: false convergence (8)}}\begin{verbatim}
## 
## Iteration 2 - deviance = 688.3464 - criterion = 0.09867735
\end{verbatim}


{\ttfamily\noindent\color{warningcolor}{\#\# Warning: Inner iterations did not coverge - nlminb message: false convergence (8)}}\begin{verbatim}
## 
## Iteration 3 - deviance = 615.7054 - criterion = 0.04444854
\end{verbatim}


{\ttfamily\noindent\color{warningcolor}{\#\# Warning: Inner iterations did not coverge - nlminb message: false convergence (8)}}\begin{verbatim}
## 
## Iteration 4 - deviance = 580.2161 - criterion = 0.0228836
\end{verbatim}


{\ttfamily\noindent\color{warningcolor}{\#\# Warning: Inner iterations did not coverge - nlminb message: false convergence (8)}}\begin{verbatim}
## 
## Iteration 5 - deviance = 561.5089 - criterion = 0.01010364
\end{verbatim}


{\ttfamily\noindent\color{warningcolor}{\#\# Warning: Inner iterations did not coverge - nlminb message: false convergence (8)}}\begin{verbatim}
## 
## Iteration 6 - deviance = 550.7341 - criterion = 0.00167807
\end{verbatim}


{\ttfamily\noindent\color{warningcolor}{\#\# Warning: Inner iterations did not coverge - nlminb message: false convergence (8)}}\begin{verbatim}
## 
## Iteration 7 - deviance = 548.1399 - criterion = 0.0004822583
\end{verbatim}


{\ttfamily\noindent\color{warningcolor}{\#\# Warning: Inner iterations did not coverge - nlminb message: false convergence (8)}}\begin{verbatim}
## 
## Iteration 8 - deviance = 547.9013 - criterion = 0.0004590341
\end{verbatim}


{\ttfamily\noindent\color{warningcolor}{\#\# Warning: Inner iterations did not coverge - nlminb message: false convergence (8)}}\begin{verbatim}
## 
## Iteration 9 - deviance = 547.8931 - criterion = 0.0004543126
\end{verbatim}


{\ttfamily\noindent\color{warningcolor}{\#\# Warning: Inner iterations did not coverge - nlminb message: false convergence (8)}}\begin{verbatim}
## 
## Iteration 10 - deviance = 547.8911 - criterion = 0.0004495207
\end{verbatim}


{\ttfamily\noindent\color{warningcolor}{\#\# Warning: Inner iterations did not coverge - nlminb message: false convergence (8)}}\begin{verbatim}
## 
## Iteration 11 - deviance = 547.8904 - criterion = 0.0004445154
\end{verbatim}


{\ttfamily\noindent\color{warningcolor}{\#\# Warning: Inner iterations did not coverge - nlminb message: false convergence (8)}}\begin{verbatim}
## 
## Iteration 12 - deviance = 547.8901 - criterion = 0.000439264
\end{verbatim}


{\ttfamily\noindent\color{warningcolor}{\#\# Warning: Inner iterations did not coverge - nlminb message: false convergence (8)}}\begin{verbatim}
## 
## Iteration 13 - deviance = 547.89 - criterion = 0.0004337693
\end{verbatim}


{\ttfamily\noindent\color{warningcolor}{\#\# Warning: Inner iterations did not coverge - nlminb message: false convergence (8)}}\begin{verbatim}
## 
## Iteration 14 - deviance = 547.89 - criterion = 0.0004280476
\end{verbatim}


{\ttfamily\noindent\color{warningcolor}{\#\# Warning: Inner iterations did not coverge - nlminb message: false convergence (8)}}\begin{verbatim}
## 
## Iteration 15 - deviance = 547.8899 - criterion = 0.0004221196
\end{verbatim}


{\ttfamily\noindent\color{warningcolor}{\#\# Warning: Inner iterations did not coverge - nlminb message: false convergence (8)}}\begin{verbatim}
## 
## Iteration 16 - deviance = 547.8899 - criterion = 0.0004160077
\end{verbatim}


{\ttfamily\noindent\color{warningcolor}{\#\# Warning: Inner iterations did not coverge - nlminb message: false convergence (8)}}\begin{verbatim}
## 
## Iteration 17 - deviance = 547.8899 - criterion = 0.0004097343
\end{verbatim}


{\ttfamily\noindent\color{warningcolor}{\#\# Warning: Inner iterations did not coverge - nlminb message: false convergence (8)}}\begin{verbatim}
## 
## Iteration 18 - deviance = 547.8899 - criterion = 0.0004033218
\end{verbatim}


{\ttfamily\noindent\color{warningcolor}{\#\# Warning: Inner iterations did not coverge - nlminb message: false convergence (8)}}\begin{verbatim}
## 
## Iteration 19 - deviance = 547.8899 - criterion = 0.0003967919
\end{verbatim}


{\ttfamily\noindent\color{warningcolor}{\#\# Warning: Inner iterations did not coverge - nlminb message: false convergence (8)}}\begin{verbatim}
## 
## Iteration 20 - deviance = 547.8899 - criterion = 0.0003901653
\end{verbatim}


{\ttfamily\noindent\color{warningcolor}{\#\# Warning: Inner iterations did not coverge - nlminb message: false convergence (8)}}\begin{verbatim}
## 
## Iteration 21 - deviance = 547.8899 - criterion = 0.0003834621
\end{verbatim}


{\ttfamily\noindent\color{warningcolor}{\#\# Warning: Inner iterations did not coverge - nlminb message: false convergence (8)}}\begin{verbatim}
## 
## Iteration 22 - deviance = 547.8899 - criterion = 0.0003767013
\end{verbatim}


{\ttfamily\noindent\color{warningcolor}{\#\# Warning: Inner iterations did not coverge - nlminb message: false convergence (8)}}\begin{verbatim}
## 
## Iteration 23 - deviance = 547.8899 - criterion = 0.0003699006
\end{verbatim}


{\ttfamily\noindent\color{warningcolor}{\#\# Warning: Inner iterations did not coverge - nlminb message: false convergence (8)}}\begin{verbatim}
## 
## Iteration 24 - deviance = 547.8899 - criterion = 0.0003630768
\end{verbatim}


{\ttfamily\noindent\color{warningcolor}{\#\# Warning: Inner iterations did not coverge - nlminb message: false convergence (8)}}\begin{verbatim}
## 
## Iteration 25 - deviance = 547.8899 - criterion = 0.0003562506
\end{verbatim}


{\ttfamily\noindent\color{warningcolor}{\#\# Warning: Algorithm did not converge}}\end{kframe}
\end{knitrout}
  \begin{frame}
    \titlepage
    {\scriptsize Code available at \url{https://osf.io/sy7uq/}.}
  \end{frame}

  \section{Objectif}

\begin{frame}[t]{Objectif\Logo}
  De mieux comprendre la façon dont la race \emph{et l'ethnicité} sont exprimées par les pronoms sujets en français louisianais, en concevant \alert{de manière générale} la langue
  \begin{enumerate}
    \item Tracy, un Créole noir:
    \begin{itemize}
      \item Elle sait dit: <<to trappé li encore>>.
      \item[] \gloss{elle disait: <<tu l'as encore trappé>>}
    \end{itemize}
    \item Gene, un Créole sans race:
    \begin{itemize}
      \item Tout ça yé va dit il faut tu prends ça.
      \item[] \gloss{Tout ce qu'ils vont dire, il faut que tu prennes ça.}
    \end{itemize}
    \item Elizabeth, une Cadienne blanche:
    \begin{itemize}
      \item Li questionne les 'tits filles qui sur le homecoming court ça yé veut.
      \item[] \gloss{Elle demande aux petites filles qui sont sur la cour de homecoming ce qu'elles veulent.}
    \end{itemize}
  \end{enumerate}
\end{frame}
  
  \section{Introduction}
  % Are Creole and French separate?
    % Socially, yes, but structurally, hardly

\begin{frame}{Le créole vs le français\Logo}
  \small
  \only<1-2>{
    Des indications de décreolisation:
    \begin{itemize}
      \item Le déterminant prenominal \lexi{les} remplace presque le postnominal \lexi{-yé} \smallcite{\parencite{mayeux_rethinking_2019}}
      \item Le modal [pe] \gloss{pouvoir} remplace presque le mot [kapab] \gloss{capable} \smallcite{\parencite{mayeux_rethinking_2019}}
      \begin{itemize}
        \item {}(Également les cas dans mes données)
      \end{itemize}
    \end{itemize}
    \uncover<2->{
      Des parlers qui mélangent les traits soi-disant créoles et français:
      \begin{itemize}
        \item {}<<Je serais kouri avec toi si je se gen [se gɛ̃] le temps>> \smallcite{\parencite[p.~361]{klingler_probleme_2005}}
      \end{itemize}
    }
  }
  \only<3->{
    Des traits partagés:
    \begin{itemize}
      \item La syntaxe dans les négations des impératifs \smallcite{\parencite{baronian_influence_2005}}
      \item {}[te] est une prononciation facile à trouver
      \item[] \lexi{il était} et \lexi{j'étais} comme [i te] et [ʃ te] (Brian 4m27 et 11m05)
      \item {}[ape] est utilisé par les locuteurs orientés vers le français
      \item[] \lexi{j'étais après guetter le TV} comme [ʃ t ape gete l ti vi] (Alice 17m40)
    \end{itemize}
    Enfin, les créolophones mêmes nomment parfois leur langue <<le français>> \smallcite{\parencite{dajko_sociolinguistics_2012, klingler_language_2003}}
  }
\end{frame}
  % What is the racial and ethnic situation in this area  

\begin{frame}[t]{La race en Louisiane\Logo}
  \small
  La race est étroitement liée aux identités cadien.ne et créole \\
  \vspace{0.25cm}
  \begin{columns}[t]
    \column{0.5\textwidth}
      Les Créoles:
      \begin{itemize}
        \item[] \emph{Période coloniale}
        \begin{itemize}
          \item Les Français.es et les Espagnol.e.s nés en Louisiane (c.-à.-d. des Blanc.he.s) \smallcite{\parencite{fortier_french_1884}}
        \end{itemize}
        \item[] \emph{le 19e siècle}
        \begin{itemize}
          \item[$\to$] Les gens de couleur libres \smallcite{\parencite{kein_placage_2000}}
        \end{itemize}
      \end{itemize}
    \column{0.5\textwidth}
      \uncover<2->{
        Les Cadien.ne.s:
        \begin{itemize}
          \item[] \emph{Historiquement}
          \begin{itemize}
            \item Des gens francophones, catholiques et ruraux.les avec des racines acadiennes \smallcite{\parencite{johnson_louisiana_1976, neumann_creole_1985, smith_influence_1939}}
          \end{itemize}
        \end{itemize}
      }
  \end{columns}
  \begin{center}
    \only<-2>{Un système racial à trois niveaux: \\ les gens en esclavage > les gens de couleur libres > les Blanc.he.s}
  \end{center}
  \begin{columns}
    \column{0.5\textwidth}
      \only<3->{
        \begin{itemize}
          \item[] \emph{Aujourd'hui}
          \begin{itemize}
            \item[$\to$] Les Noir.e.s \smallcite{\parencite{susberry_racial_2004}}
          \end{itemize}
        \end{itemize}
      }
    \column{0.5\textwidth}
      \only<3->{
        \begin{itemize}
          \item[] \emph{Aujourd'hui}
          \begin{itemize}
            \item[$\to$] Les Blanc.he.s \smallcite{\parencite{giancarlo_dont_2019}}
          \end{itemize}
        \end{itemize}
      }
  \end{columns}
  \begin{center}
    \only<3->{\alert{Un système racial binaire}}
  \end{center}
\end{frame}
  % Is this expressed in Creole? in French?

\begin{frame}{La race dans les études du créole précédentes\Logo}
  \small
  \only<-2>{
    On sait déjà qu'il y a parfois des liens entre la race et le \emph{créole} louisianais:
    \begin{itemize}
      \item Le parler des Blancs est plutôt décréolisé vers le français louisianais \smallcite{\parencite{neumann_creole_1985}}
    \end{itemize}
    \uncover<2>{
      Par exemple, pour les Blanc.he.s:
      \begin{enumerate}
        \item Peu de déterminants post-nominal au pluriel (c.-à-d. \lexi{-yé}) \smallcite{\parencite{klingler_if_2003, neumann_creole_1985}}
        \item Plus souvent les voyelles antérieures comme arrondies \smallcite{\parencite{klingler_if_2003, mayeux_rethinking_2019}}
        \item Plus de \lexi{de} entre les noms \smallcite{\parencite{klingler_if_2003}}
        \item Plus de verbes sans [e] final au présent habituel \smallcite{\parencite{klingler_if_2003}}
        \item Moins d'agglutination nominale (\lexi{un gros \alert{du}feu}) \smallcite{\parencite{mayeux_rethinking_2019}}
      \end{enumerate}
    }
  }
  \only<3->{
    Et les pronoms...
    \begin{itemize}
      \item Les Blanc.he.s produisent des pronoms plus francisés (\lexi{avec elle} et pas \lexi{avec li (f.)}) \smallcite{\parencite{klingler_if_2003}}
      \begin{itemize}
        \item {}(Pour \lexi{elle}, ce n'est plus un cas clair aujourd'hui \smallcite{\parencite{mayeux_rethinking_2019}})
      \end{itemize}
    \end{itemize}
    Et ces différences sont parfois des points de division explicite:
    \begin{itemize}
      \item Les Blanc.he.s emploient moins d'affriquées [tʃ] et [dʒ] parce qu'ils les qualifient comme <<n***e>> \smallcite{\parencite[p.~91]{neumann_creole_1985}}
      \item On peut toujours entendre l'appelation <<N-word French>>
    \end{itemize}
  }
\end{frame}

\begin{frame}{La race dans les études du français précédentes\Logo}
  Pas much...
  \begin{itemize}
    \item Des participant.e.s ont tou.te.s jugé le parler des francophones blanc.he.s et noir.e.s comme parlant cadien
    \item Les créolophones blanc.he.s était plus difficiles (19\% comme parlant cadien) \smallcite{\parencite{landry_black_2007}}
  \end{itemize}
  \uncover<2->{
    Mais l'ethnicité et la structure linguistique:
    \begin{itemize}
      \item Significative entre les Indien.ne.s et les Cadien.ne.s, y compris pour les pronoms de la 3pl \smallcite{\parencite{dajko_ethnic_2009, rottet_language_1995}}
      \item Moins de \lexi{ça} (3pl) avant les modaux \smallcite{\parencite{brown_pronominal_1988}}
    \end{itemize}
  }
\end{frame}
  \section{Question de recherche}

\begin{frame}{Question de recherche\Logo}
  \begin{itemize}
    \item[QR:] \RQone
  \end{itemize}
\end{frame}

  \section{Méthodes}

\begin{frame}{Collecte de données\Logo}
  \begin{columns}
    \column{0.5\textwidth}
      \begin{center}
        \only<1>{

\includegraphics[width=\maxwidth]{figure/unnamed-chunk-27-1} 

        }
        \only<2>{

\includegraphics[width=\maxwidth]{figure/unnamed-chunk-28-1} 

        }
        \only<3->{

\includegraphics[width=\maxwidth]{figure/unnamed-chunk-29-1} 

        }
        {\scriptsize Acadiana selon HCR 496 \smallcite{\parencite[1971, cité dans][]{trepanier_french_1988}}}
      \end{center}
    \column{0.5\textwidth}
      \only<1-3>{
        \begin{itemize}
          \item 31 interviews entre février et août 2023
          \item<2-> Où les participant.e.s ont été élevé.e.s
          \item<3-> Où les pariticpant.e.s résident aujourd'hui
        \end{itemize}
      }
      \only<4->{
        \scriptsize
        \begin{tabular}{ l r r }
          \hline
          Ethnicité    & Créole                                                                & Cadien.ne \\
                       & 11       & 20 \\
          \hline
          Genre        & Femme                                                                 & Homme \\
                       & 13           & 18 \\
          \hline
          Âge          & \multicolumn{2}{c}{24-95 ans} \\
                       &                                                                       & \\
          \hline
          Formation    & Pas d'université                                                      & Université \\
                       & 6   & 25 \\
          \hline
          Profession   & Col bleu                                                              & Col blanc \\
                       & 8 & 23 \\
          \hline
          Antécédents  & À la maison                                                           & Plus tard \\
          francophones & 23 & 3 \\
          \hline
        \end{tabular}
      }
  \end{columns}
\end{frame}

\begin{frame}{Race et ethnicité\Logo}
  \begin{columns}
    \column{0.4\textwidth}
      \scriptsize
      \begin{center}
        \begin{tabular}{l r}
          \hline
          Identification          & Compte \\
          \hline
          Noir.e singulier.ère    & 5 \\
          Créole protéiforme      & 3 \\
          Transcendant.e          & 1 \\
          Blanc.he singulier.ère  & 14 \\
          Cadien.ne singulier.ère & 3 \\
          Cadien.ne protéiforme   & 2 \\
          \hline
        \end{tabular}
        Le codage de la race basé sur \smallcite{\parencite{rockquemore_race_1999}}
      \end{center}
    \column{0.6\textwidth}
      \begin{center}

\includegraphics[width=\maxwidth]{figure/unnamed-chunk-34-1} 

      \end{center}
  \end{columns}
\end{frame}

\begin{frame}{Variantes du pronom 3pl\Logo}
  \begin{center}
  \begin{tabular}{c c c c c c c}
           &    & [i]   & & & & \\
           &    & [il]  & & & & \\
    {}[j]  &    & [ilz] & & & & \\
    {}[je] &    & [iz]  & & & & \\
    $\big\downarrow$ &    & $\big\downarrow$ & & & & \\
    yé & ça & ils & \only<1>{elles}\only<2->{\sout{elles}} & \only<1>{eux}\only<2->{\sout{eux}} & \only<1>{eux-autres}\only<2->{\sout{eux-autres}} & $N$ \\
    419 &
    238 &
    383 &
    4 &
    2 &
    1 &
    1048 \\
    \uncover<2->{
      $\big\downarrow$ & $\big\downarrow$ & $\big\downarrow$ & & & & $\big\downarrow$ \\
      yé               & \textcolor<3->{darkred}{ça}               & \textcolor<3->{darkred}{ils}              & & & & $N$ \\
      419 &
      \textcolor<3->{darkred}{238} &
      \textcolor<3->{darkred}{383} & & & &
      1040 \\
    }
  \end{tabular} \\
  \uncover<3->{\textcolor{darkred}{Typiquement produits ensemble}}
  \end{center}
\end{frame}

\begin{frame}{Le modèle\Logo}
\footnotesize
  \begin{center}
    Modèle de régression logistique multinomiale

    \begin{tabular}{p{3cm} p{3cm} p{3cm}}
                                                                                                       &                                                                                                    & \\
      \hline
      Pronom \hfill \~{}                                                                               & Predicat \hfill +                                                                                  & Ethnicité \\
      \hline
      \textbf{ça}                                                                                      & \textbf{Verbe lexical}                                                                             & \textbf{Créole} \\
      ils                                                                                              & Verbe modal                                                                                        & Cadien.ne \\
      yé                                                                                               & Verbe auxiliaire                                                                                   & \\
                                                                                                       & Préposition                                                                                        & \\
                                                                                                       & Adjectif                                                                                           & \\
                                                                                                       & & \\
      \hline
      (Intersection)                                                                                   & $N$                                                                                                & Effets aléatoires \\
      \hline
                                                                                                       &                                                                                                    & Participant.e \\
                                                                                                       &                                                                                                    & Prédicat suivant \\
    \end{tabular}
  \end{center}
\end{frame}
  
  \section{Résultats}

\begin{frame}{\lexi{ils} contre \lexi{ça}\Logo}
\footnotesize
  \begin{center}
    Modèle de régression logistique multinomiale pour \lexi{ils} contre \lexi{ça}

    \begin{tabular}{p{3cm} p{3cm} p{3cm}}
                                                                                                       &                                                                                                    & \\
      \hline
      Pronom \hfill \~{}                                                                               & Predicat \hfill +                                                                                  & Ethnicité \\
      \hline
      \textbf{ça}                                                                                      & \textbf{Verbe lexical}                                                                             & \textbf{Créole} \\
      ils                                                                                              & Verbe modal \hfill -0.07               & Cadien.ne \hfill -1.32 \\
                                                                                                       & Verbe auxiliaire \hfill 2.25      & \\
                                                                                                       & Préposition \hfill 1.25         & \\
                                                                                                       & Adjectif \hfill -23.83              & \\
                                                                                                       & & \\
      \hline
      (Intersection) \hfill -0.24            & $N = 1040$                                               & Effets aléatoires \\
      \hline
                                                                                                       &                                                                                                    & Participant.e \\
                                                                                                       &                                                                                                    & Prédicat suivant \\
    \end{tabular}
  \end{center}
\end{frame}

\begin{frame}{\lexi{yé} contre \lexi{ça}\Logo}
\footnotesize
  \begin{center}
    Modèle de régression logistique multinomiale pour \lexi{yé} contre \lexi{ça}

    \begin{tabular}{p{3cm} p{3cm} p{3cm}}
                                                                                                       &                                                                                                    & \\
      \hline
      Pronom \hfill \~{}                                                                               & Predicat \hfill +                                                                                  & Ethnicité \\
      \hline
      \textbf{ça}                                                                                      & \textbf{Verbe lexical}                                                                             & \textbf{Créole} \\
      yé                                                                                               & Verbe modal \hfill 0.91                & Cadien.ne \hfill -5.76 \\
                                                                                                       & Verbe auxiliaire \hfill 1.57       & \\
                                                                                                       & Préposition \hfill 23.66          & \\
                                                                                                       & Adjectif \hfill -2.59               & \\
                                                                                                       & & \\
      \hline
      (Intersection) \hfill 1.06             & $N = 1040$                                               & Effets aléatoires \\
      \hline
                                                                                                       &                                                                                                    & Participant.e \\
                                                                                                       &                                                                                                    & Prédicat suivant \\
    \end{tabular}
  \end{center}
\end{frame}
  
  \section{Conclusions}
  %   <<child = "./present_frames/answers.Rnw">>=
  %   @

\begin{frame}{\Logo}
  
\end{frame}
    % Why do we care about race in LF? > fractal recursivity primer
  
  \section{References}
    \printbibliography
  
  % \section{Bonus Slides}
  %   <<child = "./present_frames/bonus_use.Rnw">>=
  %   @
\end{document}
